% !TeX spellcheck = en_GB

%%%%%%%%%%%%%%%%%%%%%%%%%%%%%%%%%%%%%%%%%%
%                                        %
%    Engineer thesis LaTeX template      % 
%  compliant with the SZJK regulations   %
%                                        %
%%%%%%%%%%%%%%%%%%%%%%%%%%%%%%%%%%%%%%%%%%
%                                        %
%  (c) Krzysztof Simiński, 2018-2022     %
%                                        %
%%%%%%%%%%%%%%%%%%%%%%%%%%%%%%%%%%%%%%%%%%
%                                        %
% The latest version of the templates is %
% available at                           %
% github.com/ksiminski/polsl-aei-theses  %
%                                        %
%%%%%%%%%%%%%%%%%%%%%%%%%%%%%%%%%%%%%%%%%%
%
%
% This LaTeX project formats the final thesis 
% with compliance to the SZJK regulations. 
% Please to not change formatting (fonts, margins,
% bolds, italics, etc).
%
% You can compile the project in several ways.
%
% 1. pdfLaTeX compilation
% 
% If you use the minted package for code snippets
% compile the project like this:
%
% pdflatex -shell-escape main
% bibtex                 main
% pdflatex -shell-escape main
% pdflatex -shell-escape main
%
% If you do not use the minted package, just comment 
% the package import \usepackage{minted} and its use in the thesis.
% Then run the compilation:
%
% pdflatex main
% bibtex   main
% pdflatex main
% pdflatex main
%
%
% 2. XeLaTeX compilation
%
% Compilation with the XeLaTeX engine inserts Calibri font
% in the title page. Of course the font has to be installed.
% As above you can use the minted package or not. 
% Compilation with the minted package:
%
% xelatex -shell-escape main
% bibtex                main
% xelatex -shell-escape main
% xelatex -shell-escape main
%
% Without the minted package the compilation is simpler:
%
% xelatex main
% bibtex  main
% xelatex main
% xelatex main
%
%
% manuals for code snippets packages:
% https://ctan.org/pkg/minted
% https://ctan.org/pkg/listings
%
%%%%%%%%%%%%%%%%%%%%%%%%%%%%%%%%%%%%%%%%%%%%%%%%%%%%%%%%%%%%%%
% If you have any questions, remarks, just send me an email: %
%            krzysztof.siminski(at)polsl.pl                  %
%%%%%%%%%%%%%%%%%%%%%%%%%%%%%%%%%%%%%%%%%%%%%%%%%%%%%%%%%%%%%%

%%%%%%%%%%%%%%%%%%%%%%%%%%%%%%%%%%%%%%%%%%%%%%%%%%%%%%%%%%%%%%%%%%%%%%%%%

%%%%%%%%%%%%%%%%%%%%%%%%%%%%%%%%%%%%%%%%%%%%%%%
%                                             %
% SETTINGS                                    %
% DO NOT MODIFY!                              %
%                                             %
%%%%%%%%%%%%%%%%%%%%%%%%%%%%%%%%%%%%%%%%%%%%%%%






\documentclass[a4paper,twoside,12pt]{book}
\usepackage[utf8]{inputenc}                                      
\usepackage[T1]{fontenc}  
\usepackage{amsmath,amsfonts,amssymb,amsthm}
\usepackage[polish,british]{babel} 
\usepackage{indentfirst}



\usepackage{ifxetex}

\ifxetex
	\usepackage{fontspec}
	\defaultfontfeatures{Mapping=tex—text} % to support TeX conventions like ``——-''
	\usepackage{xunicode} % Unicode support for LaTeX character names (accents, European chars, etc)
	\usepackage{xltxtra} % Extra customizations for XeLaTeX
\else
	\usepackage{lmodern}
\fi



\usepackage[margin=2.5cm]{geometry}
\usepackage{graphicx} 
\usepackage{hyperref}
\usepackage{booktabs}
\usepackage{tikz}
\usepackage{pgfplots}
\usepackage{mathtools}
\usepackage{geometry}
\usepackage[page]{appendix} % toc,




\usepackage{csquotes}
\usepackage[style=numeric,backend=bibtex]{biblatex} % biblatex
\bibliography{biblio}              % biblatex

\usepackage{ifmtarg}   % empty commands  

\usepackage{setspace}
\onehalfspacing


\frenchspacing

%%%%%%%%%%%%%%%%%%%%%%%%%%%%%%%%%%%%%%%%%%%%%%%%%%%%%%%%%%%%%%%%%%%%%
% listings
% packages: listings or minted
% % % % % % % % % % % % % % % % % % % % % % % % % % % % % % % % % % % 

% package listings
\usepackage{listings}
\lstset{%
language=C++,%
commentstyle=\textit,%
identifierstyle=\textsf,%
keywordstyle=\sffamily\bfseries, %\texttt, %
%captionpos=b,%
tabsize=3,%
frame=lines,%
numbers=left,%
numberstyle=\tiny,%
numbersep=5pt,%
breaklines=true,%
%morekeywords={descriptor_gaussian,descriptor,partition,fcm_possibilistic,dataset,my_exception,exception,std,vector},%
escapeinside={@*}{*@},% 
}

% % % % % % % % % % % % % % % % % % % % % % % % % % % % % % % % % % % 
% package minted
%\usepackage{minted}

% This package requires a special command line option in compilation
% pdflatex -shell-escape thesis

%%%%%%%%%%%%%%%%%%%%%%%%%%%%%%%%%%%%%%%%%%%%%%%%%%%%%%%%%%%%%%%%%%%%%


%%%% TODO LIST GENERATOR %%%%%%%%%

\usepackage{color}
\definecolor{brickred}      {cmyk}{0   , 0.89, 0.94, 0.28}

\makeatletter \newcommand \kslistofremarks{\section*{Remarks} \@starttoc{rks}}
  \newcommand\l@uwagas[2]
    {\par\noindent \textbf{#2:} %\parbox{10cm}
{#1}\par} \makeatother


\newcommand{\ksremark}[1]{%
{%\marginpar{\textdbend}
{\color{brickred}{[#1]}}}%
\addcontentsline{rks}{uwagas}{\protect{#1}}%
}










%%%%%%%%%%%%%% END OF TODO LIST GENERATOR %%%%%%%%%%%  

%%%%%%%%%%%% FANCY HEADERS %%%%%%%%%%%%%%%
% brak kapitalizacji zywej paginy
\usepackage{fancyhdr}
\pagestyle{fancy}
\fancyhf{}
\fancyhead[LO]{\nouppercase{\it\rightmark}}
\fancyhead[RE]{\nouppercase{\it\leftmark}}
\fancyhead[LE,RO]{\it\thepage}


\fancypagestyle{onlyPageNumbers}{%
   \fancyhf{} 
   \fancyhead[LE,RO]{\it\thepage}
}

\fancypagestyle{noNumbers}{%
   \fancyhf{} 
   \fancyhead[LE,RO]{}
}


\fancypagestyle{PageNumbersChapterTitles}{%
   \fancyhf{} 
   \fancyhead[LO]{\nouppercase{\Firstnames \Surname}}
   \fancyhead[RE]{\nouppercase{\leftmark}} 
   \fancyfoot[CE, CO]{\thepage}
}
 



%%%%%%%%%%%%%%%%%%%%%%%%%%%







\newcounter{pagesWithoutNumbers}

%%%%%%%%%%%%%%%%%%%%%%%%%%% 
\usepackage{xstring}
\usepackage{ifthen}
\newcommand{\printOpiekun}[1]{%		

    \StrLen{\Consultant}[\mystringlen]
    \ifthenelse{\mystringlen > 0}%
    {%
       {\large{\bfseries CONSULTANT}\par}
       
       {\large{\bfseries \Consultant}\par}
    }%
    {}
} 
%
%%%%%%%%%%%%%%%%%%%%%%%%%%%%%%%%%%%%%%%%%%%%%%
 
% Please do not modify the lines below!
\author{\Firstnames\ \Surname}
\newcommand{\Author}{\Firstnames\ \MakeUppercase{\Surname}}
\newcommand{\Type}{FINAL PROJECT}
\newcommand{\Faculty}{Faculty of Automatic Control, Electronics and Computer Science}
\newcommand{\Polsl}{Silesian University of Technology}
\newcommand{\Logo}{politechnika_sl_logo_bw_pion_en.pdf}
\newcommand{\LeftId}{Student identification number}
\newcommand{\LeftProgram}{Programme}
\newcommand{\LeftSpecialisation}{Specialisation}
\newcommand{\LeftSUPERVISOR}{SUPERVISOR}
\newcommand{\LeftDEPARTMENT}{DEPARTMENT}

%%%%%%%%%%%%%%%%%%%%%%%%%%%%%%%%%%%%%%%%%%%%%%



%%%%%%%%%%%%%%%%%%%%%%%%%%%%%%%%%%%%%%%%%%%%%%%
%                                             %
% END OF SETTINGS                             %
%                                             %
%%%%%%%%%%%%%%%%%%%%%%%%%%%%%%%%%%%%%%%%%%%%%%%

% Add keywords for code snippets if you need
\lstset{%
morekeywords={string,exception,std,vector},% keyword for the listings package
}
%%%%%%%%%%%%%%%%%%%%%%%%%%%%%%%%%%%%%%%%%%%%%%%
%                                             %
% CUSTOMISE YOUR THESIS                       %
%                                             %
%%%%%%%%%%%%%%%%%%%%%%%%%%%%%%%%%%%%%%%%%%%%%%%
% TODO
\newcommand{\Firstnames}{First Names}
\newcommand{\Surname}{Surname}
\newcommand{\Supervisor}{$\langle$title first name surname$\rangle$}
\newcommand{\Title}{Thesis title in English}
\newcommand{\TitlePL}{Thesis title in Polish}
\newcommand{\Program}{Control, Electronic, and Information Engineering}
\newcommand{\Specialisation}{…}
\newcommand{\Id}{$\langle$your student id$\rangle$}
\newcommand{\Departament}{$\langle$put departament name$\rangle$}


% If you have a consultant for your thesis, put their name below ...
\newcommand{\Consultant}{$\langle$title first name surname$\rangle$}
% ... else leave the braces empty:
%\newcommand{\consultant}{} % no consultant

% end of thesis customisation
%%%%%%%%%%%%%%%%%%%%%%%%%%%%%%%%%%%%%%%%%%




%%%%%%%%%%%%%%%%%%%%%%%%%%%%%%%%%%%%%%%%%%%%%%%
%                                             %
% END OF CUSTOMISATION                        %
%                                             %
%%%%%%%%%%%%%%%%%%%%%%%%%%%%%%%%%%%%%%%%%%%%%%%

 

\begin{document}
%\kslistofremarks 

%%%%%%%%%%%%%%%%%%%%%%%%%%%%%%%%%%%%%%%%%%%%%%%
%                                             %
% TITLE PAGE                                  %
% DO NOT MODIFY!                              %
%                                             %
%%%%%%%%%%%%%%%%%%%%%%%%%%%%%%%%%%%%%%%%%%%%%%%

\pagestyle{empty}
{
	\newgeometry{top=1.5cm,%
	             bottom=2.5cm,%
	             left=3cm,
	             right=2.5cm}
 
	\ifxetex 
	  \begingroup
	  \setsansfont{Calibri}
	   
	\fi 
	 \sffamily
	\begin{center}
	\includegraphics[width=50mm]{\Logo}
	 
	
	{\Large\bfseries\Type\par}
	
	\vfill  \vfill  
			 
	{\large\Title\par}
	
	\vfill  
		
	{\large\bfseries\Author\par}
	
	{\normalsize\bfseries \LeftId: \Id}
	
	\vfill  		
 
	{\large{\bfseries \LeftProgram:} \Program\par} 
	
	{\large{\bfseries \LeftSpecialisation:} \Specialisation\par} 
	 		
	\vfill  \vfill 	\vfill 	\vfill 	\vfill 	\vfill 	\vfill  
	 
	{\large{\bfseries \LeftSUPERVISOR}\par}
	
	{\large{\bfseries \Supervisor}\par}
				
	{\large{\bfseries \LeftDEPARTMENT\ \Departament} \par}
		
	{\large{\bfseries \Faculty}\par}
		
	\vfill  \vfill  

    	
    \printOpiekun{\Consultant}
    
	\vfill  \vfill  
		
    {\large\bfseries  Gliwice \the\year}

   \end{center}	
       \ifxetex 
       	  \endgroup
       \fi
	\restoregeometry
}
  
%%%%%%%%%%%%%%%%%%%%%%%%%%%%%%%%%%%%%%%%%%%%%%%
%                                             %
% END OF TITLE PAGE                           %
%                                             %
%%%%%%%%%%%%%%%%%%%%%%%%%%%%%%%%%%%%%%%%%%%%%%%

\cleardoublepage
 
\rmfamily\normalfont
\pagestyle{empty}

  

% TODO
\subsubsection*{Thesis title} \Title

\subsubsection*{Abstract}  
(Thesis abstract – to be copied into an appropriate field during an electronic submission – in English.)

\subsubsection*{Keywords} 
(2-5 keywords, separated with commas)

\subsubsection*{Tytuł pracy} 
\begin{otherlanguage}{polish}
\TitlePL
\end{otherlanguage}

\subsubsection*{Streszczenie} 
\begin{otherlanguage}{polish}
(Thesis abstract – to be copied into an appropriate field during an electronic submission – in Polish.)
\end{otherlanguage}
\subsubsection*{Słowa kluczowe}  
\begin{otherlanguage}{polish}
(2-5 keywords, separated by commas, in Polish)
\end{otherlanguage}


%%%%%%%%%%%%%%%%%% Table of contents %%%%%%%%%%%%%%%%%%%%%%
%\pagenumbering{Roman}
\thispagestyle{empty}
\tableofcontents
\thispagestyle{empty}

%%%%%%%%%%%%%%%%%%%%%%%%%%%%%%%%%%%%%%%%%%%%%%%%%%%%%
\setcounter{pagesWithoutNumbers}{\value{page}}
\mainmatter
\pagestyle{empty}
 
\cleardoublepage

\pagestyle{PageNumbersChapterTitles}

%%%%%%%%%%%%%% body of the thesis %%%%%%%%%%%%%%%%%

% TODO
\chapter{Introduction}

\begin{itemize}
\item introduction into the problem domain
\item settling of the problem in the domain
\item objective of the thesis 
\item scope of the thesis
\item short description of chapters
\item clear description of contribution of the thesis's author – in case of more authors table with enumeration of contribution of authors
\end{itemize}
% TODO
\chapter{[Problem analysis]}

\begin{itemize}
\item  problem analysis
\item state of the art, problem statement
\item  literature research (all sources in the thesis have to be referenced \cite{bib:article,bib:book,bib:conference,bib:internet})
\item description of existing solutions (also scientific ones, if the problem is scientifically researched), algorithms,  location of the thesis in the scientific domain
\end{itemize}



Mathematical formulae  
\begin{align}
y = \frac{\partial x}{\partial t}
\end{align}
and single math symbols $x$ and $y$ are typeset in the mathematical mode.

% TODO
\chapter{Requirements and tools}

\begin{itemize}
\item functional and nonfunctional requirements
\item use cases (UML diagrams)
\item description of tools
\item methodology of design and implementation
\end{itemize} 
% TODO
\chapter{External specification}
\begin{itemize}
\item hardware and software requirements
\item installation procedure
\item activation procedure
\item types of users
\item user manual
\item system administration
\item security issues
\item example of usage
\item working scenarios (with screenshots or output files)
\end{itemize}

 


 
\begin{figure}
\centering
\begin{tikzpicture}
\begin{axis}[
    y tick label style={
        /pgf/number format/.cd,
            fixed,    
            fixed zerofill, % 1.0 zamiast 1
            precision=1,
        /tikz/.cd
    },
    x tick label style={
        /pgf/number format/.cd,
            fixed,
            fixed zerofill,
            precision=2,
        /tikz/.cd
    }
]
\addplot [domain=0.0:0.1] {rnd};
\end{axis} 
\end{tikzpicture}
\caption{Figure caption (below the figure).}
\label{fig:2}
\end{figure}

% TODO
\chapter{Internal specification}

\begin{itemize}
\item concept of the system
\item system architecture
\item description of data structures (and data bases)
\item components, modules, libraries, resume of important classes (if used)
\item resume of important algorithms (if used)
\item details of implementation of selected parts
\item applied design patterns
\item UML diagrams
\end{itemize}



% % % % % % % % % % % % % % % % % % % % % % % % % % % % % % % % % % % 
% To use the minted packages uncomment the package import in        %
% file config/settings.tex :  \usepackage{minted}                   %
% and compile with the shell escape                                 %
% pdflatex -shell-escape main                                       %
% % % % % % % % % % % % % % % % % % % % % % % % % % % % % % % % % % % 


Use special environments for inline code, eg  \lstinline|int a;| (package \texttt{listings})% or  \mintinline{C++}|int a;| (package \texttt{minted})
. Longer parts of code put in the figure environment, eg. code in Fig. \ref{fig:pseudocode:listings}% and Fig. \ref{fig:pseudocode:minted}
. Very long listings–move to an appendix.


\clearpage
\begin{figure}
\centering
\begin{lstlisting}
class test : public basic
{
    public:
      test (int a);
      friend std::ostream operator<<(std::ostream & s, 
                                     const test & t);
    protected:
      int _a;  
      
};
\end{lstlisting}
\caption{Pseudocode in \texttt{listings}.}
\label{fig:pseudocode:listings}
\end{figure}

%\begin{figure}
%\centering
%\begin{minted}[linenos,frame=lines]{c++}
%class test : public basic
%{
%    public:
%      test (int a);
%      friend std::ostream operator<<(std::ostream & s, 
%                                     const test & t);
%    protected:
%      int _a;  
%      
%};
%\end{minted}
%\caption{Pseudocode in \texttt{minted}.}
%\label{fig:pseudocode:minted}
%\end{figure}


% TODO
\chapter{Verification and validation}
\begin{itemize}
\item testing paradigm (eg V model)
\item test cases, testing scope (full / partial)
\item detected and fixed bugs
\item results of experiments (optional)
\end{itemize}

 
\begin{table}
\centering
\caption{A caption of a table is \textbf{above} it.}
\label{id:tab:wyniki}
\begin{tabular}{rrrrrrrr}
\toprule
	         &                                     \multicolumn{7}{c}{method}                                      \\
	         \cmidrule{2-8}
	         &         &         &        \multicolumn{3}{c}{alg. 3}        & \multicolumn{2}{c}{alg. 4, $\gamma = 2$} \\
	         \cmidrule(r){4-6}\cmidrule(r){7-8}
	$\zeta$ &     alg. 1 &   alg. 2 & $\alpha= 1.5$ & $\alpha= 2$ & $\alpha= 3$ &   $\beta = 0.1$  &   $\beta = -0.1$ \\
\midrule
	       0 &  8.3250 & 1.45305 &       7.5791 &    14.8517 &    20.0028 & 1.16396 &                       1.1365 \\
	       5 &  0.6111 & 2.27126 &       6.9952 &    13.8560 &    18.6064 & 1.18659 &                       1.1630 \\
	      10 & 11.6126 & 2.69218 &       6.2520 &    12.5202 &    16.8278 & 1.23180 &                       1.2045 \\
	      15 &  0.5665 & 2.95046 &       5.7753 &    11.4588 &    15.4837 & 1.25131 &                       1.2614 \\
	      20 & 15.8728 & 3.07225 &       5.3071 &    10.3935 &    13.8738 & 1.25307 &                       1.2217 \\
	      25 &  0.9791 & 3.19034 &       5.4575 &     9.9533 &    13.0721 & 1.27104 &                       1.2640 \\
	      30 &  2.0228 & 3.27474 &       5.7461 &     9.7164 &    12.2637 & 1.33404 &                       1.3209 \\
	      35 & 13.4210 & 3.36086 &       6.6735 &    10.0442 &    12.0270 & 1.35385 &                       1.3059 \\
	      40 & 13.2226 & 3.36420 &       7.7248 &    10.4495 &    12.0379 & 1.34919 &                       1.2768 \\
	      45 & 12.8445 & 3.47436 &       8.5539 &    10.8552 &    12.2773 & 1.42303 &                       1.4362 \\
	      50 & 12.9245 & 3.58228 &       9.2702 &    11.2183 &    12.3990 & 1.40922 &                       1.3724 \\
\bottomrule
\end{tabular}
\end{table}  

% TODO
\chapter{Conclusions}
\begin{itemize}
\item achieved results with regard to objectives of the thesis and requirements
\item path of further development (eg functional extension …)
\item encountered difficulties and problems
\end{itemize}

\backmatter 

%\bibliographystyle{plain} % bibtex
%\bibliography{biblio}     % bibtex
\printbibliography           % biblatex 
\addcontentsline{toc}{chapter}{References}

\begin{appendices}

% TODO
\chapter{Index of abbreviations and symbols}

\begin{itemize}
\item[DNA] deoxyribonucleic acid
\item[MVC] model--view--controller 
\item[$N$] cardinality of data set
\item[$\mu$] membership function of a fuzzy set
\item[$\mathbb{E}$] set of edges of a graph
\item[$\mathcal{L}$] Laplace transformation
\end{itemize}
% TODO
\chapter{Listings}

(Put long listings here.)

\begin{lstlisting}
if (_nClusters < 1)
	throw std::string ("unknown number of clusters");
if (_nIterations < 1 and _epsilon < 0)
	throw std::string ("You should set a maximal number of iteration or minimal difference -- epsilon.");
if (_nIterations > 0 and _epsilon > 0)
	throw std::string ("Both number of iterations and minimal epsilon set -- you should set either number of iterations or minimal epsilon.");
\end{lstlisting}


% % % % % % % % % % % % % % % % % % % % % % % % % % % % % % % % % % % 
% To use the minted packages uncomment the package import in        %
% file config/settings.tex :  \usepackage{minted}                   %
% and compile with the shell escape                                 %
% pdflatex -shell-escape main                                       %
% % % % % % % % % % % % % % % % % % % % % % % % % % % % % % % % % % % 

%\begin{minted}[linenos,breaklines,frame=lines]{c++}
%if (_nClusters < 1)
%	throw std::string ("unknown number of clusters");
%if (_nIterations < 1 and _epsilon < 0)
%	throw std::string ("You should set a maximal number of iteration or minimal difference -- epsilon.");
%if (_nIterations > 0 and _epsilon > 0)
%	throw std::string ("Both number of iterations and minimal epsilon set -- you should set either number of iterations or minimal epsilon.");
%\end{minted} 
% TODO
\chapter{List of additional files in~electronic submission (if applicable)}


Additional files uploaded to the system include:
\begin{itemize}
\item source code of the application,
\item test data,
\item a video file showing how software or hardware developed for thesis is used,
\item etc.
\end{itemize}
\listoffigures
\addcontentsline{toc}{chapter}{List of figures}
\listoftables
\addcontentsline{toc}{chapter}{List of tables}
	
\end{appendices}

\end{document}


%% Finis coronat opus.

