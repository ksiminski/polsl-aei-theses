%%%%%%%%%%%%%%%%%%%%%%%%%%%%%%%%%%%%%%%%%%
%                                        %
% Szablon pracy dyplomowej magisterskiej % 
%                                        %
%%%%%%%%%%%%%%%%%%%%%%%%%%%%%%%%%%%%%%%%%%
%  (c) Krzysztof Simiński, 2018-2021     %
%%%%%%%%%%%%%%%%%%%%%%%%%%%%%%%%%%%%%%%%%%

% kompilacja:

% pdflatex thesis
% biber    thesis
% pdflatex thesis
% pdflatex thesis 

%%%%%%%%%%%%%%%%%%%%%%%%%%%%%%%%%%%%%%%%%%


\documentclass[a4paper,twoside,12pt]{book}
\usepackage[utf8]{inputenc}                                      
\usepackage[T1]{fontenc}  
\usepackage{amsmath,amsfonts,amssymb,amsthm}
\usepackage[british,polish]{babel} 
\usepackage{indentfirst}
\usepackage{lmodern}
\usepackage{graphicx} 
\usepackage{hyperref}
\usepackage{booktabs}
%\usepackage{tikz}
%\usepackage{pgfplots}
\usepackage{mathtools}
\usepackage{geometry}
\usepackage[page]{appendix} % toc,
\renewcommand{\appendixtocname}{Dodatki}
\renewcommand{\appendixpagename}{Dodatki}
\renewcommand{\appendixname}{Dodatek}

\usepackage{csquotes}
\usepackage[natbib=true]{biblatex}
\bibliography{bibliografia}

\usepackage{setspace}
\onehalfspacing


\frenchspacing

\usepackage{listings}
\lstset{
	language={},
	basicstyle=\ttfamily,
	keywordstyle=\lst@ifdisplaystyle\color{blue}\fi,
	commentstyle=\color{gray}
}

%%%%%%%%%

%%%% TODO LIST GENERATOR %%%%%%%%%

%\usepackage{tikz}
%\usepackage{manfnt}   % dangerous sign 
\usepackage{color}
\definecolor{brickred}      {cmyk}{0   , 0.89, 0.94, 0.28}

\makeatletter \newcommand \kslistofremarks{\section*{Uwagi} \@starttoc{rks}}
  \newcommand\l@uwagas[2]
    {\par\noindent \textbf{#2:} %\parbox{10cm}
{#1}\par} \makeatother


\newcommand{\ksremark}[1]{%
{%\marginpar{\textdbend}
{\color{brickred}{[#1]}}}%
\addcontentsline{rks}{uwagas}{\protect{#1}}%
}

\newcommand{\comma}{\ksremark{przecinek}}
\newcommand{\nocomma}{\ksremark{bez przecinka}}
\newcommand{\styl}{\ksremark{styl}}
\newcommand{\ortografia}{\ksremark{ortografia}}
\newcommand{\fleksja}{\ksremark{fleksja}}
\newcommand{\pauza}{\ksremark{pauza `--', nie dywiz `-'}}
\newcommand{\kolokwializm}{\ksremark{kolokwializm}}
\newcommand{\cudzyslowy}{\ksremark{,,polskie cudzysłowy''}}

%%%%%%%%%%%%%% END OF TODO LIST GENERATOR %%%%%%%%%%%

%%%%%%%%%%%% ZYWA PAGINA %%%%%%%%%%%%%%%
% brak kapitalizacji zywej paginy
\usepackage{fancyhdr}
\pagestyle{fancy}
\fancyhf{}
\fancyhead[LO]{\nouppercase{\it\rightmark}}
\fancyhead[RE]{\nouppercase{\it\leftmark}}
\fancyhead[LE,RO]{\it\thepage}


\fancypagestyle{tylkoNumeryStron}{%
   \fancyhf{} 
   \fancyhead[LE,RO]{\it\thepage}
}

\fancypagestyle{NumeryStronNazwyRozdzialow}{%
   \fancyhf{} 
   \fancyhead[LO]{\nouppercase{\it\rightmark}}
   \fancyhead[RE]{\nouppercase{\it\leftmark}}
   \fancyhead[LE,RO]{\it\thepage}
}


%%%%%%%%%%%%% OBCE WTRETY  
\newcommand{\obcy}[1]{\emph{#1}}
\newcommand{\ang}[1]{{\selectlanguage{british}\obcy{#1}}}
%%%%%%%%%%%%%%%%%%%%%%%%%%%%%

% polskie oznaczenia funkcji matematycznych
\renewcommand{\tan}{\operatorname {tg}}
\renewcommand{\log}{\operatorname {lg}}

% jeszcze jakies drobiazgi

\newcounter{stronyPozaNumeracja}

\newcommand{\hcancel}[1]{%
    \tikz[baseline=(tocancel.base)]{
        \node[inner sep=0pt,outer sep=0pt] (tocancel) {#1};
        \draw[red] (tocancel.south west) -- (tocancel.north east);
    }%
}%

\newcommand{\miesiac}{%
  \ifcase\the\month
  \or styczeń% 1
  \or luty% 2
  \or marzec% 3
  \or kwiecień% 4
  \or maj% 5
  \or czerwiec% 6
  \or lipiec% 7
  \or sierpień% 8
  \or wrzesień% 9
  \or październik% 10
  \or listopad% 11
  \or grudzień% 12
  \fi}


%%%%%%%%%%%%%%%%%%%%%%%%%%%%%%%%%%%%%%%%%%%%%%
% Helvetica font macros for the title page:
\newcommand{\headerfont}{\fontfamily{phv}\fontsize{18}{18}\bfseries\scshape\selectfont}
\newcommand{\titlefont}{\fontfamily{phv}\fontsize{18}{18}\selectfont}
\newcommand{\otherfont}{\fontfamily{phv}\fontsize{14}{14}\selectfont}

%%%%%%%%%%%%%%%%%%%%%%%%%%%%%%%%%%%%%%%%%%%%%%
%%%%%%%%%%%%%%%%%%%%%%%%%%%%%%%%%%%%%%%%%%%%%%
%%%%%%%%%%%%%%%%%%%%%%%%%%%%%%%%%%%%%%%%%%%%%%
%%%%%%%%%%%%%%%%%%%%%%%%%%%%%%%%%%%%%%%%%%%%%%
%%%%%%%%%%%%%%%%%%%%%%%%%%%%%%%%%%%%%%%%%%%%%%
%%%%%%%%%%%%%%%%%%%%%%%%%%%%%%%%%%%%%%%%%%%%%%
%%%%%%%%%%%%%%%%%%%%%%%%%%%%%%%%%%%%%%%%%%%%%%


\newcommand{\autor}{Imię Nazwisko}
\newcommand{\promotor}{tytuł/stopień naukowy Imię Nazwisko}
\newcommand{\konsultant}{tytuł/stopień naukowy Imię Nazwisko}
\newcommand{\tytul}{Tytuł pracy dyplomowej magisterskiej}
\newcommand{\polsl}{Politechnika Śląska}
\newcommand{\wydzial}{Wydział Automatyki, Elektroniki i Informatyki}
\newcommand{\kierunek}{Kierunek: [wpisać właściwy]}

\begin{document}
%\kslistofremarks 
	
%%%%%%%%%%%%%%%%%%  STRONA TYTULOWA %%%%%%%%%%%%%%%%%%%
\pagestyle{empty}
{
	\newgeometry{top=2.5cm,%
	             bottom=2.5cm,%
	             left=3cm,
	             right=2.5cm}
	\sffamily
	\rule{0cm}{0cm}
	
	\begin{center}
	\includegraphics[width=45mm]{logo_pl.jpg}
	\end{center} 
	\vspace{1cm}
	\begin{center}
	\headerfont \polsl
	\end{center}
	\begin{center}
	\headerfont \wydzial
	\end{center}
	\begin{center}
	\headerfont \kierunek
	\end{center}
	\vfill
	\begin{center}
	\titlefont Praca dyplomowa magisterska
	\end{center}
	\vfill
	
	\begin{center}
	\otherfont \tytul\par
	\end{center}
	
	\vfill
	
	\vfill
	 
	\noindent\vbox
	{
		\hbox{\otherfont Autor: \autor}
		\vspace{12pt}
		\hbox{\otherfont Promotor: \promotor}
		\vspace{12pt}
		\hbox{\otherfont Konsultant: \konsultant}
	}
	\vfill 
 
   \begin{center}
   \otherfont Gliwice,  \miesiac\ \the\year
   \end{center}	
	\restoregeometry
}
  

\cleardoublepage
 

\rmfamily
\normalfont



%%%%%%%%%%%%%%%%%% SPIS TRESCI %%%%%%%%%%%%%%%%%%%%%%
\pagenumbering{Roman}
\pagestyle{tylkoNumeryStron}
\tableofcontents

%%%%%%%%%%%%%%%%%%%%%%%%%%%%%%%%%%%%%%%%%%%%%%%%%%%%%
\setcounter{stronyPozaNumeracja}{\value{page}}
\mainmatter


\pagestyle{empty}

\chapter*{Streszczenie}
\addcontentsline{toc}{chapter}{Streszczenie}

Odpowiednie pole w systemie APD powinno zawierać kopię tego streszczenia. Streszczenie wraz ze słowami kluczowymi nie powinno przekroczyć jednej strony.

\paragraph{Słowa kluczowe:} 2-5 slow (fraz) kluczowych, oddzielonych przecinkami


\cleardoublepage

\pagestyle{NumeryStronNazwyRozdzialow}
%%%%%%%%%%%%%% wlasciwa tresc pracy %%%%%%%%%%%%%%%%%


\chapter{Wstęp}

\begin{itemize}
\item wprowadzenie w problem/zagadnienie 
\item osadzenie problemu w dziedzinie 
\item cel pracy 
\item zakres pracy 
\item zwięzła charakterystyka rozdziałów 
\end{itemize}


\chapter{[Analiza tematu]}


\begin{itemize}
\item analiza tematu
\item wprowadzenie do dziedziny (\ang{state of the art}) – sformułowanie problemu, 
\item poszerzone studia literaturowe, przegląd literatury tematu (należy wskazać źródła wszystkich informacji zawartych w pracy)
\item opis znanych rozwiązań, algorytmów, osadzenie pracy w kontekście
\item Tytuł rozdziału jest często zbliżony do tematu pracy. 
\item Rozdział jest wysycony cytowaniami do literatury \cite{bib:artykul,bib:ksiazka,bib:konferencja}. 
Cytowanie książki \cite{bib:ksiazka}, artykułu w czasopiśmie \cite{bib:artykul}, artykułu konferencyjnego \cite{bib:konferencja} lub strony internetowej \cite{bib:internet}.
\end{itemize}




\chapter{[Przedmiot pracy]}

\begin{itemize}
\item rozwiązanie zaproponowane przez dyplomanta
\item analiza teoretyczna rozwiązania
\item uzasadnienie wyboru zastosowanych metod, algorytmów, narzędzi
\end{itemize}

\chapter{Badania}

 

Rozdział przedstawia przeprowadzone badania. Jest to zasadnicza część i~musi wyraźnie dominować w~pracy.
Badania i analizę wyników należy przeprowadzić, tak jak jest przyjęte w środowisku naukowym (na przykład korzystanie z danych benchmarkowych, walidacja krzyżowa, zapewnienie powtarzalności testów itd). 

\section{Metodyka badań}

\begin{itemize}
\item opis metodyki badań
\item opis stanowiska badawczego (opis interfejsu aplikacji badawczych -- w~załączniku)
\end{itemize}


\section{Zbiory danych}

\begin{itemize}
\item opis danych
\end{itemize}


\section{Wyniki}

\begin{itemize}
\item prezentacja wyników, opracowanie i poszerzona dyskusja  wyników, wnioski
\end{itemize}


\begin{figure}
\centering
\includegraphics[width=6cm]{logo_pl.jpg}
\caption{Podpis rysunku po rysunkiem.}
\label{fig:2}
\end{figure}

\begin{table}
\centering
\caption{Opis tabeli nad nią.}
\label{id:tab:wyniki}
\begin{tabular}{rrrrrrrr}
\toprule
	         &                                     \multicolumn{7}{c}{metoda}                                      \\
	         \cmidrule{2-8}
	         &         &         &        \multicolumn{3}{c}{alg. 3}        & \multicolumn{2}{c}{alg. 4, $\gamma = 2$} \\
	         \cmidrule(r){4-6}\cmidrule(r){7-8}
	$\zeta$ &     alg. 1 &   alg. 2 & $\alpha= 1.5$ & $\alpha= 2$ & $\alpha= 3$ &   $\beta = 0.1$  &   $\beta = -0.1$ \\
\midrule
	       0 &  8.3250 & 1.45305 &       7.5791 &    14.8517 &    20.0028 & 1.16396 &                       1.1365 \\
	       5 &  0.6111 & 2.27126 &       6.9952 &    13.8560 &    18.6064 & 1.18659 &                       1.1630 \\
	      10 & 11.6126 & 2.69218 &       6.2520 &    12.5202 &    16.8278 & 1.23180 &                       1.2045 \\
	      15 &  0.5665 & 2.95046 &       5.7753 &    11.4588 &    15.4837 & 1.25131 &                       1.2614 \\
	      20 & 15.8728 & 3.07225 &       5.3071 &    10.3935 &    13.8738 & 1.25307 &                       1.2217 \\
	      25 &  0.9791 & 3.19034 &       5.4575 &     9.9533 &    13.0721 & 1.27104 &                       1.2640 \\
	      30 &  2.0228 & 3.27474 &       5.7461 &     9.7164 &    12.2637 & 1.33404 &                       1.3209 \\
	      35 & 13.4210 & 3.36086 &       6.6735 &    10.0442 &    12.0270 & 1.35385 &                       1.3059 \\
	      40 & 13.2226 & 3.36420 &       7.7248 &    10.4495 &    12.0379 & 1.34919 &                       1.2768 \\
	      45 & 12.8445 & 3.47436 &       8.5539 &    10.8552 &    12.2773 & 1.42303 &                       1.4362 \\
	      50 & 12.9245 & 3.58228 &       9.2702 &    11.2183 &    12.3990 & 1.40922 &                       1.3724 \\
\bottomrule
\end{tabular}
\end{table}  



\chapter{Podsumowanie}
\begin{itemize}
\item syntetyczny opis wykonanych prac
\item wnioski
\item możliwość rozwoju, kontynuacji prac, potencjalne nowe kierunki
\item Czy cel pracy zrealizowany? 
\end{itemize}


%%%%%%%%%%%%%%%%%%%%%%%%%%%%%%%%%%%%%%%%%%
\backmatter
\pagenumbering{Roman}
\stepcounter{stronyPozaNumeracja}
\setcounter{page}{\value{stronyPozaNumeracja}}

\pagestyle{tylkoNumeryStron}

%%%%%%%%%%% bibliografia %%%%%%%%%%%%
%\bibliographystyle{plplain} % bibtex
%\bibliography{bibliografia} % bibtex
\printbibliography           % biblatex 
\addcontentsline{toc}{chapter}{Bibliografia}
%%%%%%%%  DODATKI %%%%%%%%%%%%%%%%%%% 

\begin{appendices} 


\chapter*{Dokumentacja techniczna}
\addcontentsline{toc}{chapter}{Dokumentacja techniczna}

\chapter*{Spis skrótów i symboli}
\addcontentsline{toc}{chapter}{Spis skrótów i symboli}

\begin{itemize}
\item[DNA] kwas deoksyrybonukleinowy (ang. \ang{deoxyribonucleic acid})
\item[MVC] model -- widok -- kontroler (ang. \ang{model--view--controller}) 
\item[$N$] liczebność zbioru danych
\item[$\mu$] stopnień przynależności do zbioru
\item[$\mathbb{E}$] zbiór krawędzi grafu
\item[$\mathcal{L}$] transformata Laplace'a 
\end{itemize}


 

\chapter*{Lista dodatkowych plików,
uzupełniających tekst pracy} 
\addcontentsline{toc}{chapter}{Lista dodatkowych plików}

W systemie do pracy dołączono dodatkowe pliki zawierające:
\begin{itemize}
\item źródła programu,
\item zbiory danych użyte w~eksperymentach,
%\item film pokazujący działanie opracowanego oprogramowania lub zaprojektowanego i wykonanego urządzenia,
\item itp.
\end{itemize}

\listoffigures
\addcontentsline{toc}{chapter}{Spis rysunkow}
\listoftables
\addcontentsline{toc}{chapter}{Spis tabel}
	
\end{appendices}


\end{document}


%% Finis coronat opus.
