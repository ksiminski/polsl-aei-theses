% !TeX spellcheck = pl_PL
%
%%%%%%%%%%%%%%%%%%%%%%%%%%%%%%%%%%%%%%%%%%
%                                        %
% Szablon pracy dyplomowej inzynierskiej % 
% zgodny  z aktualnymi  przepisami  SZJK %
%                                        %
%%%%%%%%%%%%%%%%%%%%%%%%%%%%%%%%%%%%%%%%%%
%  (c) Krzysztof Simiński, 2018-2022     %
%%%%%%%%%%%%%%%%%%%%%%%%%%%%%%%%%%%%%%%%%%
%
%
% Projekt LaTeXowy zapewnia odpowiednie formatowanie pracy,
% zgodnie z wymaganiami Systemu zapewniania jakości kształcenia.
% Proszę nie zmieniać ustawień formatowania (np. fontu,
% marginesów, wytłuszczeń, kursywy itd. ).
%
% Projekt można kompilować na kilka sposobów.
%
% 1. kompilacja pdfLaTeX
%
% Jeżeli w pracy używany jest pakiet minted do formatowania
% kodów źródłowych należy kompilować w następujący sposób:
%
% pdflatex -shell-escape praca
% biber                  praca
% pdflatex -shell-escape praca
% pdflatex -shell-escape praca 
%
% Jeżeli pakiet minted nie jest wykorzystywany, można zakomentować
% import tego pakietu \usepackage{minted} i użycie w kodzie pracy.
% Wtedy wystarczy prosta kompilacja 
%
% pdflatex praca
% biber    praca
% pdflatex praca
% pdflatex praca 
%
%
% 2. kompilacja XeLaTeX
%
% Kompilatacja przy użyciu XeLaTeXa różni się tym, że na stronie
% tytułowej używany jest font Calibri. Wymaga to jego uprzedniego
% zainstalowania. Podobnie jak w przypadku pdfLaTeXa użycie
% pakietu minted wymaga kompilacji jak poniżej:
%
% xelatex -shell-escape praca
% biber                 praca
% xelatex -shell-escape praca
% xelatex -shell-escape praca 
%
% Bez pakietu minted kompilacja jest trochę prostsza:
%
% xelatex praca
% biber   praca
% xelatex praca
% xelatex praca 
%
%
% dokumentacja pakietów do kodów źródłowych:
% https://ctan.org/pkg/minted
% https://ctan.org/pkg/listings
%
%%%%%%%%%%%%%%%%%%%%%%%%%%%%%%%%%%%%%%%%%%%%%%%%%%%%%
% W przypadku pytań, uwag, proszę pisać na adres:   %
%      krzysztof.siminski(małpa)polsl.pl            %
%%%%%%%%%%%%%%%%%%%%%%%%%%%%%%%%%%%%%%%%%%%%%%%%%%%%%

%%%%%%%%%%%%%%%%%%%%%%%%%%%%%%%%%%%%%%%%%%%%%%%%%%%%%%%%%%%%%%%%%%%%%%%%%
% Proszę wpisać swoje dane w poniższych definicjach.

% TODO
\newcommand{\FirstName}{Imię}
\newcommand{\Surname}{Nazwisko}
\newcommand{\Supervisor}{$\langle$tytuł lub stopień naukowy oraz imię i nazwisko$\rangle$}
\newcommand{\Title}{Tytuł pracy dyplomowej inżynierskiej}
\newcommand{\TitleAlt}{Thesis title in English}
\newcommand{\Program}{$\langle$wpisać właściwy$\rangle$}
\newcommand{\Specialisation}{$\langle$wpisać właściwą$\rangle$}
\newcommand{\Id}{$\langle$wpisać właściwy$\rangle$}
\newcommand{\Departament}{$\langle$wpisać właściwą$\rangle$}

% Jeżeli został wyznaczony promotor pomocniczy lub opieku proszę go/ją wpisać ...
\newcommand{\opiekun}{$\langle$stopień naukowy imię i nazwisko$\rangle$}
% ... w przeciwnym razie proszę zostawić puste miejsce jak poniżej:
%\newcommand{\opiekun}{} % brak promotowa pomocniczego / opiekuna

% koniec fragmentu do modyfikacji
%%%%%%%%%%%%%%%%%%%%%%%%%%%%%%%%%%%%%%%%%%


%%%%%%%%%%%%%%%%%%%%%%%%%%%%%%%%%%%%%%%%%%%%%%%
%                                             %
%   PLEASE DO NOT MODIFY THE SETTINGS BELOW!  %
%                                             %
%%%%%%%%%%%%%%%%%%%%%%%%%%%%%%%%%%%%%%%%%%%%%%%



\documentclass[a4paper,twoside,12pt]{book}
\usepackage[utf8]{inputenc}                                      
\usepackage[T1]{fontenc}  
\usepackage{amsmath,amsfonts,amssymb,amsthm}
\usepackage[polish,british]{babel} 
\usepackage{indentfirst}
\usepackage{xurl}
\usepackage{xstring}
\usepackage{ifthen}



\usepackage{ifxetex}

\ifxetex
	\usepackage{fontspec}
	\defaultfontfeatures{Mapping=tex—text} % to support TeX conventions like ``——-''
	\usepackage{xunicode} % Unicode support for LaTeX character names (accents, European chars, etc)
	\usepackage{xltxtra} % Extra customizations for XeLaTeX
\else
	\usepackage{lmodern}
\fi



\usepackage[margin=2.5cm]{geometry}
\usepackage{graphicx} 
\usepackage{hyperref}
\usepackage{booktabs}
\usepackage{tikz}
\usepackage{pgfplots}
\usepackage{mathtools}
\usepackage{geometry}
\usepackage{subcaption}   % subfigures
\usepackage[page]{appendix} % toc,




\usepackage{csquotes}
\usepackage[natbib=true,backend=bibtex,maxbibnames=99]{biblatex}  % compilation of bibliography with BibTeX
%\usepackage[natbib=true,backend=biber,maxbibnames=99]{biblatex}  % compilation of bibliography with Biber
\bibliography{biblio/biblio}

\usepackage{ifmtarg}   % empty commands  

\usepackage{setspace}
\onehalfspacing


\frenchspacing



%%%% TODO LIST GENERATOR %%%%%%%%%

\usepackage{color}
\definecolor{brickred}      {cmyk}{0   , 0.89, 0.94, 0.28}

\makeatletter \newcommand \kslistofremarks{\section*{Remarks} \@starttoc{rks}}
  \newcommand\l@uwagas[2]
    {\par\noindent \textbf{#2:} %\parbox{10cm}
{#1}\par} \makeatother


\newcommand{\ksremark}[1]{%
{%\marginpar{\textdbend}
{\color{brickred}{[#1]}}}%
\addcontentsline{rks}{uwagas}{\protect{#1}}%
}










%%%%%%%%%%%%%% END OF TODO LIST GENERATOR %%%%%%%%%%%  

\newcommand{\printCoauthor}{%		
    \StrLen{\FirstNameCoauthor}[\FNCoALen]
    \ifthenelse{\FNCoALen > 0}%
    {%
		{\large\bfseries\Coauthor\par}
	
		{\normalsize\bfseries \LeftId: \IdCoauthor\par}
    }%
    {}
} 

%%%%%%%%%%%%%%%%%%%%%
\newcommand{\autor}{%		
    \StrLen{\FirstNameCoauthor}[\FNCoALenXX]
    \ifthenelse{\FNCoALenXX > 0}%
    {\FirstNameAuthor\ \SurnameAuthor, \FirstNameCoauthor\ \SurnameCoauthor}%
	{\FirstNameAuthor\ \SurnameAuthor}%
}
%%%%%%%%%%%%%%%%%%%%%

\StrLen{\FirstNameCoauthor}[\FNCoALen]
\ifthenelse{\FNCoALen > 0}%
{%
\author{\FirstNameAuthor\ \SurnameAuthor, \FirstNameCoauthor\ \SurnameCoauthor}
}%
{%
\author{\FirstNameAuthor\ \SurnameAuthor}
}%

%%%%%%%%%%%% FANCY HEADERS %%%%%%%%%%%%%%%
% no capitalisation of headers
\usepackage{fancyhdr}
\pagestyle{fancy}
\fancyhf{}
\fancyhead[LO]{\nouppercase{\it\rightmark}}
\fancyhead[RE]{\nouppercase{\it\leftmark}}
\fancyhead[LE,RO]{\it\thepage}


\fancypagestyle{onlyPageNumbers}{%
   \fancyhf{} 
   \fancyhead[LE,RO]{\it\thepage}
}

\fancypagestyle{noNumbers}{%
   \fancyhf{} 
   \fancyhead[LE,RO]{}
}


\fancypagestyle{PageNumbersChapterTitles}{%
   \fancyhf{} 
   \fancyhead[LE]{\nouppercase{\autor}}
   \fancyhead[RO]{\nouppercase{\leftmark}} 
   \fancyfoot[CE, CO]{\thepage}
}
 



%%%%%%%%%%%%%%%%%%%%%%%%%%%








\newcounter{pagesWithoutNumbers}

%%%%%%%%%%%%%%%%%%%%%%%%%%% 
\newcommand{\printOpiekun}[1]{%		

    \StrLen{\Consultant}[\mystringlen]
    \ifthenelse{\mystringlen > 0}%
    {%
       {\large{\bfseries CONSULTANT}\par}
       
       {\large{\bfseries \Consultant}\par}
    }%
    {}
} 
%
%%%%%%%%%%%%%%%%%%%%%%%%%%%%%%%%%%%%%%%%%%%%%%
 
% Please do not modify the lines below!
\newcommand{\Author}{\FirstNameAuthor\ \MakeUppercase{\SurnameAuthor}} 
\newcommand{\Coauthor}{\FirstNameCoauthor\ \MakeUppercase{\SurnameCoauthor}}
\newcommand{\Type}{FINAL PROJECT}
\newcommand{\Faculty}{Faculty of Automatic Control, Electronics and Computer Science}
\newcommand{\Polsl}{Silesian University of Technology}
\newcommand{\Logo}{graf/politechnika_sl_logo_bw_pion_en.pdf}
\newcommand{\LeftId}{Student identification number}
\newcommand{\LeftProgram}{Programme}
\newcommand{\LeftSpecialisation}{Specialisation}
\newcommand{\LeftSUPERVISOR}{SUPERVISOR}
\newcommand{\LeftDEPARTMENT}{DEPARTMENT}
%%%%%%%%%%%%%%%%%%%%%%%%%%%%%%%%%%%%%%%%%%%%%%

%%%%%%%%%%%%%%%%%%%%%%%%%%%%%%%%%%%%%%%%%%%%%%%
%                                             %
% END OF SETTINGS                             %
%                                             %
%%%%%%%%%%%%%%%%%%%%%%%%%%%%%%%%%%%%%%%%%%%%%%%

 % Proszę nie modyfikować pliku settings.tex

% Poniżej można dodać słowa kluczowe dla
\lstset{%
morekeywords={string,exception,std,vector},% słowa kluczowe rozpoznawane przez pakiet listings
}


%%%%%%%%%%%%%%%%%%%%%%%%%%%%%%%%%%%%%%%%   


\begin{document}
%\kslistofremarks 

%%%%%%%%%%%%%%%%%%%%%%%%%%%%%%%%%%%%%%%%%%%%%%%
%                                             %
% PROSZĘ NIE MODYFIKOWAĆ PONIŻSZYCH USTAWIEŃ! %
%                                             %
%%%%%%%%%%%%%%%%%%%%%%%%%%%%%%%%%%%%%%%%%%%%%%%


%%%%%%%%%%%%%%%%%%  STRONA TYTULOWA %%%%%%%%%%%%%%%%%%%
\pagestyle{empty}
{
	\newgeometry{top=1.5cm,%
	             bottom=4cm,%
	             left=3cm,
	             right=2.5cm}
 
	\ifxetex 
	  \begingroup
	  \setsansfont{Calibri}
	   
	\fi 
	 \sffamily
	\begin{center}
	\includegraphics[width=50mm]{logo_pl.jpg}
	 
	
	{\Large\bfseries\typ\par}
	
	\vfill  \vfill  
			 
	{\large\tytul\par}
	
	\vfill  
		
	{\large\bfseries\autor\par}
	
	{\normalsize\bfseries Nr albumu: \album}
	
	\vfill  		
 
	{\large{\bfseries Kierunek:} \kierunek\par} 
	
	{\large{\bfseries Specjalność:} \specjalnosc\par} 
	 		
	\vfill  \vfill 	\vfill 	\vfill 	\vfill 	\vfill 	\vfill  
	 
	{\large{\bfseries PROWADZĄCY PRACĘ}\par}
	
	{\large{\bfseries \promotor}\par}
				
	{\large{\bfseries KATEDRA \katedra} \par}
		
	{\large{\bfseries \wydzial}\par}
		
	\vfill  \vfill  

    	
    \printOpiekun{\opiekun}
    
	\vfill  \vfill  
		
    {\large\bfseries  Gliwice \the\year}

   \end{center}	
       \ifxetex 
       	  \endgroup
       \fi
	\restoregeometry
}
  
  % Proszę nie modyfikować pliku titlepage.tex

\cleardoublepage
 
\rmfamily\normalfont
\pagestyle{empty}

  
%%% No to zaczynamy pisać pracę :-) %%%%

% TODO
\subsubsection*{Thesis title} \Title

\subsubsection*{Abstract}  
(Thesis abstract – to be copied into an appropriate field during an electronic submission – in English.)

\subsubsection*{Keywords} 
(2-5 keywords, separated with commas)

\subsubsection*{Tytuł pracy} 
\begin{otherlanguage}{polish}
\TitleAlt
\end{otherlanguage}

\subsubsection*{Streszczenie} 
\begin{otherlanguage}{polish}
(Thesis abstract – to be copied into an appropriate field during an electronic submission – in Polish.)
\end{otherlanguage}
\subsubsection*{Słowa kluczowe}  
\begin{otherlanguage}{polish}
(2-5 keywords, separated by commas, in Polish)
\end{otherlanguage}

 % informacje redakcyjne


%%%%%%%%%%%%%%%%%% SPIS TRESCI %%%%%%%%%%%%%%%%%%%%%%
%\pagenumbering{Roman}
\thispagestyle{empty}
\tableofcontents
\thispagestyle{empty}

%%%%%%%%%%%%%%%%%%%%%%%%%%%%%%%%%%%%%%%%%%%%%%%%%%%%%
\setcounter{stronyPozaNumeracja}{\value{page}}
\mainmatter
\pagestyle{empty}
 
\cleardoublepage

\pagestyle{NumeryStronNazwyRozdzialow}

%%%%%%%%%%%%%% wlasciwa tresc pracy %%%%%%%%%%%%%%%%%

% TODO
\chapter{Introduction}

%\begin{itemize}
%\item introduction into the problem domain
%\item settling of the problem in the domain
%\item objective of the thesis 
%\item scope of the thesis
%\item short description of chapters
%\item clear description of contribution of the thesis's author
%\end{itemize}

  % wstęp

% TODO
\chapter{[Analiza tematu]}


\begin{itemize}
\item analiza tematu
\item wprowadzenie do dziedziny (\ang{state of the art}) – sformułowanie problemu, 
\item poszerzone studia literaturowe, przegląd literatury tematu (należy wskazać źródła wszystkich informacji zawartych w pracy)
\item opis znanych rozwiązań, algorytmów, osadzenie pracy w kontekście
\item Tytuł rozdziału jest często zbliżony do tematu pracy. 
\item Rozdział jest wysycony cytowaniami do literatury \cite{bib:artykul,bib:ksiazka,bib:konferencja}. 
Cytowanie książki \cite{bib:ksiazka}, artykułu w czasopiśmie \cite{bib:artykul}, artykułu konferencyjnego \cite{bib:konferencja} lub strony internetowej \cite{bib:internet}.
\end{itemize}
 % analiza tematu 

% TODO
\chapter{Wymagania i narzędzia}
\label{ch:wymagania-i-narzedzia}

\begin{itemize}
\item wymagania funkcjonalne i niefunkcjonalne
\item przypadki użycia (diagramy UML) -- dla prac, w których mają zastosowanie
\item opis narzędzi, metod eksperymentalnych, metod modelowania itp.
\item metodyka pracy nad projektowaniem i implementacją -- dla prac, w których ma to zastosowanie
\end{itemize}
 % Wymagania i narzędzia

% TODO
\chapter{External specification}
\begin{itemize}
\item hardware and software requirements
\item installation procedure
\item activation procedure
\item types of users
\item user manual
\item system administration
\item security issues
\item example of usage
\item working scenarios (with screenshots or output files)
\end{itemize}

 


 
\begin{figure}
\centering
\begin{tikzpicture}
\begin{axis}[
    y tick label style={
        /pgf/number format/.cd,
            fixed,    
            fixed zerofill, % 1.0 zamiast 1
            precision=1,
        /tikz/.cd
    },
    x tick label style={
        /pgf/number format/.cd,
            fixed,
            fixed zerofill,
            precision=2,
        /tikz/.cd
    }
]
\addplot [domain=0.0:0.1] {rnd};
\end{axis} 
\end{tikzpicture}
\caption{Figure caption (below the figure).}
\label{fig:2}
\end{figure}

 % [Właściwy dla kierunku -- np. Specyfikacja zewnętrzna]

% TODO
\chapter{Internal specification}

\begin{itemize}
\item concept of the system
\item system architecture
\item description of data structures (and data bases)
\item components, modules, libraries, resume of important classes (if used)
\item resume of important algorithms (if used)
\item details of implementation of selected parts
\item applied design patterns
\item UML diagrams
\end{itemize}



% % % % % % % % % % % % % % % % % % % % % % % % % % % % % % % % % % % 
% To use the minted package, import the package:                    %
% \usepackage{minted}                                               %
% and compile with the shell escape                                 %
% pdflatex -shell-escape main                                       %
% % % % % % % % % % % % % % % % % % % % % % % % % % % % % % % % % % % 


Use special environments for inline code, eg  \lstinline|int a;| (package \texttt{listings})% or  \mintinline{C++}|int a;| (package \texttt{minted})
. Longer parts of code put in the figure environment, eg. code in Fig. \ref{fig:pseudocode:listings}% and Fig. \ref{fig:pseudocode:minted}
. Very long listings–move to an appendix.


\clearpage
\begin{figure}
\centering
\begin{lstlisting}
class test : public basic
{
    public:
      test (int a);
      friend std::ostream operator<<(std::ostream & s, 
                                     const test & t);
    protected:
      int _a;  
      
};
\end{lstlisting}
\caption{Pseudocode in \texttt{listings}.}
\label{fig:pseudocode:listings}
\end{figure}

%\begin{figure}
%\centering
%\begin{minted}[linenos,frame=lines]{c++}
%class test : public basic
%{
%    public:
%      test (int a);
%      friend std::ostream operator<<(std::ostream & s, 
%                                     const test & t);
%    protected:
%      int _a;  
%      
%};
%\end{minted}
%\caption{Pseudocode in \texttt{minted}.}
%\label{fig:pseudocode:minted}
%\end{figure}


 % [Właściwy dla kierunku -- np. Specyfikacja wewnętrzna]

% TODO
\chapter{Weryfikacja i walidacja}
\label{ch:06}
\begin{itemize}
\item sposób testowania w ramach pracy (np. odniesienie do modelu V)
\item organizacja eksperymentów
\item przypadki testowe zakres testowania (pełny/niepełny)
\item wykryte i usunięte błędy
\item opcjonalnie wyniki badań eksperymentalnych
\end{itemize}

\begin{table}
\centering
\caption{Nagłówek tabeli jest nad tabelą.}
\label{id:tab:wyniki}
\begin{tabular}{rrrrrrrr}
\toprule
	         &                                     \multicolumn{7}{c}{metoda}                                      \\
	         \cmidrule{2-8}
	         &         &         &        \multicolumn{3}{c}{alg. 3}        & \multicolumn{2}{c}{alg. 4, $\gamma = 2$} \\
	         \cmidrule(r){4-6}\cmidrule(r){7-8}
	$\zeta$ &     alg. 1 &   alg. 2 & $\alpha= 1.5$ & $\alpha= 2$ & $\alpha= 3$ &   $\beta = 0.1$  &   $\beta = -0.1$ \\
\midrule
	       0 &  8.3250 & 1.45305 &       7.5791 &    14.8517 &    20.0028 & 1.16396 &                       1.1365 \\
	       5 &  0.6111 & 2.27126 &       6.9952 &    13.8560 &    18.6064 & 1.18659 &                       1.1630 \\
	      10 & 11.6126 & 2.69218 &       6.2520 &    12.5202 &    16.8278 & 1.23180 &                       1.2045 \\
	      15 &  0.5665 & 2.95046 &       5.7753 &    11.4588 &    15.4837 & 1.25131 &                       1.2614 \\
	      20 & 15.8728 & 3.07225 &       5.3071 &    10.3935 &    13.8738 & 1.25307 &                       1.2217 \\
	      25 &  0.9791 & 3.19034 &       5.4575 &     9.9533 &    13.0721 & 1.27104 &                       1.2640 \\
	      30 &  2.0228 & 3.27474 &       5.7461 &     9.7164 &    12.2637 & 1.33404 &                       1.3209 \\
	      35 & 13.4210 & 3.36086 &       6.6735 &    10.0442 &    12.0270 & 1.35385 &                       1.3059 \\
	      40 & 13.2226 & 3.36420 &       7.7248 &    10.4495 &    12.0379 & 1.34919 &                       1.2768 \\
	      45 & 12.8445 & 3.47436 &       8.5539 &    10.8552 &    12.2773 & 1.42303 &                       1.4362 \\
	      50 & 12.9245 & 3.58228 &       9.2702 &    11.2183 &    12.3990 & 1.40922 &                       1.3724 \\
\bottomrule
\end{tabular}
\end{table}  

 % Weryfikacja i walidacja

% TODO
\chapter{Podsumowanie i wnioski}
\begin{itemize}
\item uzyskane wyniki w świetle postawionych celów i zdefiniowanych wyżej wymagań
\item kierunki ewentualnych danych prac (rozbudowa funkcjonalna …)
\item problemy napotkane w trakcie pracy
\end{itemize}

 % Podsumowanie i wnioski

\backmatter 

%\bibliographystyle{plplain} % bibtex
%\bibliography{bibliografia} % bibtex
\printbibliography           % biblatex 
\addcontentsline{toc}{chapter}{Bibliografia}

\begin{appendices}

% TODO
\chapter{Lista dodatkowych plików, uzupełniających tekst pracy (jeżeli dotyczy)} 

W systemie do pracy dołączono dodatkowe pliki zawierające:
\begin{itemize}
\item źródła programu,
\item zbiory danych użyte w~eksperymentach,
\item film pokazujący działanie opracowanego oprogramowania lub zaprojektowanego i wykonanego urządzenia,
\item itp.
\end{itemize}
 % Spis skrótów i symboli

% TODO
\chapter{Listings}

(Put long listings here.)

\begin{lstlisting}
if (_nClusters < 1)
	throw std::string ("unknown number of clusters");
if (_nIterations < 1 and _epsilon < 0)
	throw std::string ("You should set a maximal number of iteration or minimal difference -- epsilon.");
if (_nIterations > 0 and _epsilon > 0)
	throw std::string ("Both number of iterations and minimal epsilon set -- you should set either number of iterations or minimal epsilon.");
\end{lstlisting}


% % % % % % % % % % % % % % % % % % % % % % % % % % % % % % % % % % % 
% To use the minted packages uncomment the package import in        %
% file config/settings.tex :  \usepackage{minted}                   %
% and compile with the shell escape                                 %
% pdflatex -shell-escape main                                       %
% % % % % % % % % % % % % % % % % % % % % % % % % % % % % % % % % % % 

%\begin{minted}[linenos,breaklines,frame=lines]{c++}
%if (_nClusters < 1)
%	throw std::string ("unknown number of clusters");
%if (_nIterations < 1 and _epsilon < 0)
%	throw std::string ("You should set a maximal number of iteration or minimal difference -- epsilon.");
%if (_nIterations > 0 and _epsilon > 0)
%	throw std::string ("Both number of iterations and minimal epsilon set -- you should set either number of iterations or minimal epsilon.");
%\end{minted}  % Źródła

% TODO
\chapter{Lista dodatkowych plików, uzupełniających tekst pracy} 


W systemie do pracy dołączono dodatkowe pliki zawierające:
\begin{itemize}
\item źródła programu,
\item dane testowe,
\item film pokazujący działanie opracowanego oprogramowania lub zaprojektowanego i~wykonanego urządzenia,
\item itp.
\end{itemize}
 % Lista dodatkowych plików, uzupełniających tekst pracy

\listoffigures
\addcontentsline{toc}{chapter}{Spis rysunków}
\listoftables
\addcontentsline{toc}{chapter}{Spis tabel}
	
\end{appendices}

\end{document}


%% Finis coronat opus.
