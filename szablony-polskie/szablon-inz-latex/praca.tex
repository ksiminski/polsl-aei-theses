% !TeX spellcheck = pl_PL
%%%%%%%%%%%%%%%%%%%%%%%%%%%%%%%%%%%%%%%%%%
%                                        %
% Szablon pracy dyplomowej inzynierskiej % 
% zgodny  z aktualnymi  przepisami  SZJK %
%                                        %
%%%%%%%%%%%%%%%%%%%%%%%%%%%%%%%%%%%%%%%%%%
%  (c) Krzysztof Simiński, 2018-2022     %
%%%%%%%%%%%%%%%%%%%%%%%%%%%%%%%%%%%%%%%%%%

%% pdfLaTeX 
% kompilacja:

% pdflatex praca
% biber    praca
% pdflatex praca
% pdflatex praca 


% kompilacja z pakietem minted

% pdflatex -shell-escape praca
% biber    praca
% pdflatex -shell-escape praca
% pdflatex -shell-escape praca 


%%%%%
%% XeLaTeX (wymaga zainstalowanej czcionki Calibri użytej na stronie tytułowej)

% kompilacja:

% xelatex praca
% biber    praca
% xelatex praca
% xelatex praca 


% kompilacja z pakietem minted

% xelatex -shell-escape praca
% biber    praca
% xelatex -shell-escape praca
% xelatex -shell-escape praca 

%%%%%%%%%%%%%%%%%%%%%%%%%%%%%%%%%%%%%%%%%%
 

%%%%%%%%%%%%%%%%%%%%%%%%%%%%%%%%%%%%%%%%%%

% Proszę wpisać swoje dane w poniższych definicjach.

\newcommand{\imie}{Imię}
\newcommand{\nazwisko}{Nazwisko}
\newcommand{\promotor}{$\langle$tytuł lub stopień naukowy oraz imię i nazwisko$\rangle$}
\newcommand{\tytul}{Tytuł pracy dyplomowej inżynierskiej}
\newcommand{\Title}{Thesis title in English}
\newcommand{\kierunek}{$\langle$wpisać właściwy$\rangle$}
\newcommand{\specjalnosc}{$\langle$wpisać właściwą$\rangle$}
\newcommand{\album}{$\langle$wpisać właściwy$\rangle$}
\newcommand{\katedra}{$\langle$wpisać właściwą$\rangle$}
\newcommand{\opiekun}{$\langle$stopień naukowy imię i nazwisko$\rangle$}

% koniec fragmentu do modyfikacji

%%%%%%%%%%%%%%%%%%%%%%%%%%%%%%%%%%%%%%%%%%%%%%%
%                                             %
% PROSZĘ NIE MODYFIKOWAĆ PONIŻSZYCH USTAWIEŃ! %
%                                             %
%%%%%%%%%%%%%%%%%%%%%%%%%%%%%%%%%%%%%%%%%%%%%%%

\documentclass[a4paper,twoside,12pt]{book}
\usepackage[utf8]{inputenc}                                      
\usepackage[T1]{fontenc}  
\usepackage{amsmath,amsfonts,amssymb,amsthm}
\usepackage[british,polish]{babel} 
\usepackage{indentfirst}



\usepackage{ifxetex}

\ifxetex
	\usepackage{fontspec}
	\defaultfontfeatures{Mapping=tex—text} % to support TeX conventions like ``——-''
	\usepackage{xunicode} % Unicode support for LaTeX character names (accents, European chars, etc)
	\usepackage{xltxtra} % Extra customizations for XeLaTeX
\else
	\usepackage{lmodern}
\fi



\usepackage[margin=2.5cm]{geometry}
\usepackage{graphicx} 
\usepackage{hyperref}
\usepackage{booktabs}
\usepackage{tikz}
\usepackage{pgfplots}
\usepackage{mathtools}
\usepackage{geometry}
\usepackage[page]{appendix} % toc,
\renewcommand{\appendixtocname}{Dodatki}
\renewcommand{\appendixpagename}{Dodatki}
\renewcommand{\appendixname}{Dodatek}

\usepackage{csquotes}
\usepackage[natbib=true]{biblatex}
\bibliography{bibliografia}

\usepackage{lipsum}


\usepackage{setspace}
\onehalfspacing


\frenchspacing

%%%%%%%%%%%%%%%%%%%%%%%%%%%%%%%%%%%%%%%%%%%%%%%%%%%%%%%%%%%%%%%%%%%%%
% listingi i fragmentu kodu źródłowego 
% pakiet: listings lub minted
% % % % % % % % % % % % % % % % % % % % % % % % % % % % % % % % % % % 

% bibliotegka listings
\usepackage{listings}
\lstset{%
language=C++,%
commentstyle=\textit,%
identifierstyle=\textsf,%
keywordstyle=\sffamily\bfseries, %\texttt, %
%captionpos=b,%
tabsize=3,%
frame=lines,%
numbers=left,%
numberstyle=\tiny,%
numbersep=5pt,%
breaklines=true,%
morekeywords={descriptor_gaussian,descriptor,partition,fcm_possibilistic,dataset,my_exception,exception,std,vector},%
escapeinside={@*}{*@},%
%texcl=true, % wylacza tryb verbatim w komentarzach jednolinijkowych
}

% % % % % % % % % % % % % % % % % % % % % % % % % % % % % % % % % % % 
% pakiet minted
\usepackage{minted}

% pakiet wymaga specjalnego kompilowania:
% pdflatex -shell-escape praca

%%%%%%%%%%%%%%%%%%%%%%%%%%%%%%%%%%%%%%%%%%%%%%%%%%%%%%%%%%%%%%%%%%%%%


%%%% TODO LIST GENERATOR %%%%%%%%%

%\usepackage{tikz}
%\usepackage{manfnt}   % dangerous sign 
\usepackage{color}
\definecolor{brickred}      {cmyk}{0   , 0.89, 0.94, 0.28}

\makeatletter \newcommand \kslistofremarks{\section*{Uwagi} \@starttoc{rks}}
  \newcommand\l@uwagas[2]
    {\par\noindent \textbf{#2:} %\parbox{10cm}
{#1}\par} \makeatother


\newcommand{\ksremark}[1]{%
{%\marginpar{\textdbend}
{\color{brickred}{[#1]}}}%
\addcontentsline{rks}{uwagas}{\protect{#1}}%
}

\newcommand{\comma}{\ksremark{przecinek}}
\newcommand{\nocomma}{\ksremark{bez przecinka}}
\newcommand{\styl}{\ksremark{styl}}
\newcommand{\ortografia}{\ksremark{ortografia}}
\newcommand{\fleksja}{\ksremark{fleksja}}
\newcommand{\pauza}{\ksremark{pauza `--', nie dywiz `-'}}
\newcommand{\kolokwializm}{\ksremark{kolokwializm}}
\newcommand{\cudzyslowy}{\ksremark{,,polskie cudzysłowy''}}

%%%%%%%%%%%%%% END OF TODO LIST GENERATOR %%%%%%%%%%%

%%%%%%%%%%%% ZYWA PAGINA %%%%%%%%%%%%%%%
% brak kapitalizacji zywej paginy
\usepackage{fancyhdr}
\pagestyle{fancy}
\fancyhf{}
\fancyhead[LO]{\nouppercase{\it\rightmark}}
\fancyhead[RE]{\nouppercase{\it\leftmark}}
\fancyhead[LE,RO]{\it\thepage}


\fancypagestyle{tylkoNumeryStron}{%
   \fancyhf{} 
   \fancyhead[LE,RO]{\it\thepage}
}

\fancypagestyle{bezNumeracji}{%
   \fancyhf{} 
   \fancyhead[LE,RO]{}
}

\fancypagestyle{NumeryStronNazwyRozdzialow}{%
   \fancyhf{} 
   \fancyhead[LE]{\nouppercase{\imie\ \nazwisko}}
   \fancyhead[RO]{\nouppercase{\leftmark}}
%   \fancyhead[LE,RO]{\it\thepage}
   \fancyfoot[CE, CO]{\thepage}
}


%%%%%%%%%%%%% OBCE WTRETY  
\newcommand{\obcy}[1]{\emph{#1}}
\newcommand{\ang}[1]{{\selectlanguage{british}\obcy{#1}}}
%%%%%%%%%%%%%%%%%%%%%%%%%%%%%

% polskie oznaczenia funkcji matematycznych
\renewcommand{\tan}{\operatorname {tg}}
\renewcommand{\log}{\operatorname {lg}}

% jeszcze jakies drobiazgi

\newcounter{stronyPozaNumeracja}

 

%%%%%%%%%%%%%%%%%%%%%%%%%%%%%%%%%%%%%%%%%%%%%%
 
% Proszę nie modyfikować poniższych definicji!
\author{\imie\ \nazwisko}
\newcommand{\autor}{\imie\ \MakeUppercase{\nazwisko}}
\newcommand{\typ}{PROJEKT INŻYNIERSKI}
\newcommand{\wydzial}{Wydział Automatyki, Elektroniki i Informatyki} 

%%%%%%%%%%%%%%%%%%%%%%%%%%%%%%%%%%%%%%%%%%%%%%


\begin{document}
%\kslistofremarks 

%%%%%%%%%%%%%%%%%%  STRONA TYTULOWA %%%%%%%%%%%%%%%%%%%
\pagestyle{empty}
{
	\newgeometry{top=1.5cm,%
	             bottom=4cm,%
	             left=3cm,
	             right=2.5cm}
 
	\ifxetex 
	  \begingroup
	  \setsansfont{Calibri}
	   
	\fi 
	 \sffamily
	\begin{center}
	\includegraphics[width=50mm]{logo_pl.jpg}
	 
	
	{\Large\bfseries\typ\par}
	
	\vfill  \vfill  
			 
	{\large\tytul\par}
	
	\vfill  
		
	{\large\bfseries\autor\par}
	
	{\normalsize\bfseries Nr albumu: \album}
	
	\vfill  		
 
	{\large{\bfseries Kierunek:} \kierunek\par} 
	
	{\large{\bfseries Specjalność:} \specjalnosc\par} 
	 		
	\vfill  \vfill 	\vfill 	\vfill 	\vfill 	\vfill 	\vfill  
	 
	{\large{\bfseries PROWADZĄCY PRACĘ}\par}
	
	{\large{\bfseries \promotor}\par}
				
	{\large{\bfseries KATEDRA \katedra} \par}
		
	{\large{\bfseries \wydzial}\par}
		
	\vfill  \vfill  
					
	{\large{\bfseries OPIEKUN, PROMOTOR POMOCNICZY (jeżeli został powołany)}\par}  % Zakomentuj, jeżeli nie ma promotora pomocniczego!
		
	{\large{\bfseries \opiekun}\par} % Zakomentuj, jeżeli nie ma promotora pomocniczego!
		
	\vfill  \vfill  
		
    {\large\bfseries  Gliwice \the\year}

   \end{center}	
       \ifxetex 
       	  \endgroup
       \fi
	\restoregeometry
}
  

\cleardoublepage
 

\rmfamily
\normalfont
\pagestyle{empty}

%%%%%%%%%%%%%%%%%%%%%%%%%%%%%%%%%%%%%%%%  
%%%%%%%%%%%%%%%%%%%%%%%%%%%%%%%%%%%%%%%%  
%%%%%%%%%%%%%%%%%%%%%%%%%%%%%%%%%%%%%%%%  
%%%%%%%%%%%%%%%%%%%%%%%%%%%%%%%%%%%%%%%%  
%%%%%%%%%%%%%%%%%%%%%%%%%%%%%%%%%%%%%%%%  
%%%%%%%%%%%%%%%%%%%%%%%%%%%%%%%%%%%%%%%%
  
%%% No to zaczynamy pisać pracę :-) %%%%


\subsubsection*{Tytuł pracy} \tytul

\subsubsection*{Streszczenie}  
(Streszczenie pracy – odpowiednie pole w systemie APD powinno zawierać kopię tego streszczenia. Streszczenie wraz ze słowami kluczowymi nie powinno przekroczyć objętości jednej strony.)

\subsubsection*{Słowa kluczowe} 
2-5 slow (fraz) kluczowych, oddzielonych przecinkami

\subsubsection*{Thesis title} 
\begin{otherlanguage}{british}
\Title
\end{otherlanguage}

\subsubsection*{Abstract} 
\begin{otherlanguage}{british}
(Thesis abstract – to be copied into an appropriate field during an electronic submission. The abstract and keywords should not exceed one page.)
\end{otherlanguage}
\subsubsection*{Key words}  
\begin{otherlanguage}{british}
2-5 keywords, separated by commas
\end{otherlanguage}

 

%%%%%%%%%%%%%%%%%% SPIS TRESCI %%%%%%%%%%%%%%%%%%%%%%
%\pagenumbering{Roman}
\thispagestyle{empty}
\tableofcontents
\thispagestyle{empty}

%%%%%%%%%%%%%%%%%%%%%%%%%%%%%%%%%%%%%%%%%%%%%%%%%%%%%
\setcounter{stronyPozaNumeracja}{\value{page}}
\mainmatter
\pagestyle{empty}
 
\cleardoublepage

\pagestyle{NumeryStronNazwyRozdzialow}

%%%%%%%%%%%%%% wlasciwa tresc pracy %%%%%%%%%%%%%%%%%

\chapter{Wstęp}

\begin{itemize}
\item wprowadzenie w problem/zagadnienie
\item osadzenie problemu w dziedzinie
\item cel pracy
\item zakres pracy
\item zwięzła charakterystyka rozdziałów
\item jednoznaczne określenie wkładu autora, w przypadku prac wieloosobowych – tabela z autorstwem poszczególnych elementów pracy
\end{itemize}

 


\chapter{[Analiza tematu]}

\begin{itemize}
\item sformułowanie problemu
\item osadzenie tematu w kontekście aktualnego stanu wiedzy (\ang{state of the art}) o poruszanym problemie
\item  studia literaturowe \cite{bib:artykul,bib:ksiazka,bib:konferencja,bib:internet} -  opis znanych rozwiązań (także opisanych naukowo, jeżeli problem jest poruszany w publikacjach naukowych), algorytmów, 
\end{itemize}


Wzory  
\begin{align}
y = \frac{\partial x}{\partial t}
\end{align}
jak i pojedyncze symbole $x$ i $y$  składa się w trybie matematycznym.

\chapter{Wymagania i narzędzia}

\begin{itemize}
\item wymagania funkcjonalne i niefunkcjonalne
\item przypadki użycia (diagramy UML) -- dla prac, w których mają zastosowanie
\item opis narzędzi, metod eksperymentalnych, metod modelowania itp.
\item metodyka pracy nad projektowaniem i implementacją -- dla prac, w których ma to zastosowanie
\end{itemize}

\chapter{[Właściwy dla kierunku -- np. Specyfikacja zewnętrzna]}
Jeśli „Specyfikacja zewnętrzna”:
\begin{itemize}
\item  wymagania sprzętowe i programowe
\item  sposób instalacji
\item  sposób aktywacji
\item  kategorie użytkowników
\item  sposób obsługi
\item  administracja systemem
\item  kwestie bezpieczeństwa
\item  przykład działania
\item  scenariusze korzystania z systemu (ilustrowane zrzutami z ekranu lub generowanymi dokumentami)
\end{itemize}

 
\begin{figure}
\centering
\begin{tikzpicture}
\begin{axis}[
    y tick label style={
        /pgf/number format/.cd,
            fixed,   % po zakomentowaniu os rzednych jest indeksowana wykladniczo
            fixed zerofill, % 1.0 zamiast 1
            precision=1,
        /tikz/.cd
    },
    x tick label style={
        /pgf/number format/.cd,
            fixed,
            fixed zerofill,
            precision=2,
        /tikz/.cd
    }
]
\addplot [domain=0.0:0.1] {rnd};
\end{axis} 
\end{tikzpicture}
\caption{Podpis rysunku po rysunkiem.}
\label{fig:2}
\end{figure}


\chapter{[Właściwy dla kierunku -- np. Specyfikacja wewnętrzna]}
Jeśli „Specyfikacja wewnętrzna”:
\begin{itemize}
\item przedstawienie idei
\item architektura systemu
\item opis struktur danych (i organizacji baz danych)
\item komponenty, moduły, biblioteki, przegląd ważniejszych klas (jeśli występują)
\item przegląd ważniejszych algorytmów (jeśli występują)
\item szczegóły implementacji wybranych fragmentów, zastosowane wzorce projektowe
\item diagramy UML
\end{itemize}



Krótka wstawka kodu w linii tekstu jest możliwa, np.  \lstinline|int a;| (biblioteka \texttt{listings}) lub  \mintinline{C++}|int a;| (biblioteka \texttt{minted}). 
Dłuższe fragmenty lepiej jest umieszczać jako rysunek, np. kod na rys \ref{fig:pseudokod:listings} i rys. \ref{fig:pseudokod:minted}, a naprawdę długie fragmenty – w załączniku.
 
% dokumentacja:
% https://ctan.org/pkg/minted
% https://ctan.org/pkg/listingss


\begin{figure}
\centering
\begin{lstlisting}
class test : public basic
{
    public:
      test (int a);
      friend std::ostream operator<<(std::ostream & s, 
                                     const test & t);
    protected:
      int _a;  
      
};
\end{lstlisting}
\caption{Pseudokod w \texttt{listings}.}
\label{fig:pseudokod:listings}
\end{figure}

\begin{figure}
\centering
\begin{minted}[linenos,frame=lines]{c++}
class test : public basic
{
    public:
      test (int a);
      friend std::ostream operator<<(std::ostream & s, 
                                     const test & t);
    protected:
      int _a;  
      
};
\end{minted}
\caption{Pseudokod w \texttt{minted}.}
\label{fig:pseudokod:minted}
\end{figure}



\chapter{Weryfikacja i walidacja}
\begin{itemize}
\item sposób testowania w ramach pracy (np. odniesienie do modelu V)
\item organizacja eksperymentów
\item przypadki testowe zakres testowania (pełny/niepełny)
\item wykryte i usunięte błędy
\item opcjonalnie wyniki badań eksperymentalnych
\end{itemize}

\begin{table}
\centering
\caption{Nagłówek tabeli jest nad nią.}
\label{id:tab:wyniki}
\begin{tabular}{rrrrrrrr}
\toprule
	         &                                     \multicolumn{7}{c}{metoda}                                      \\
	         \cmidrule{2-8}
	         &         &         &        \multicolumn{3}{c}{alg. 3}        & \multicolumn{2}{c}{alg. 4, $\gamma = 2$} \\
	         \cmidrule(r){4-6}\cmidrule(r){7-8}
	$\zeta$ &     alg. 1 &   alg. 2 & $\alpha= 1.5$ & $\alpha= 2$ & $\alpha= 3$ &   $\beta = 0.1$  &   $\beta = -0.1$ \\
\midrule
	       0 &  8.3250 & 1.45305 &       7.5791 &    14.8517 &    20.0028 & 1.16396 &                       1.1365 \\
	       5 &  0.6111 & 2.27126 &       6.9952 &    13.8560 &    18.6064 & 1.18659 &                       1.1630 \\
	      10 & 11.6126 & 2.69218 &       6.2520 &    12.5202 &    16.8278 & 1.23180 &                       1.2045 \\
	      15 &  0.5665 & 2.95046 &       5.7753 &    11.4588 &    15.4837 & 1.25131 &                       1.2614 \\
	      20 & 15.8728 & 3.07225 &       5.3071 &    10.3935 &    13.8738 & 1.25307 &                       1.2217 \\
	      25 &  0.9791 & 3.19034 &       5.4575 &     9.9533 &    13.0721 & 1.27104 &                       1.2640 \\
	      30 &  2.0228 & 3.27474 &       5.7461 &     9.7164 &    12.2637 & 1.33404 &                       1.3209 \\
	      35 & 13.4210 & 3.36086 &       6.6735 &    10.0442 &    12.0270 & 1.35385 &                       1.3059 \\
	      40 & 13.2226 & 3.36420 &       7.7248 &    10.4495 &    12.0379 & 1.34919 &                       1.2768 \\
	      45 & 12.8445 & 3.47436 &       8.5539 &    10.8552 &    12.2773 & 1.42303 &                       1.4362 \\
	      50 & 12.9245 & 3.58228 &       9.2702 &    11.2183 &    12.3990 & 1.40922 &                       1.3724 \\
\bottomrule
\end{tabular}
\end{table}  

 
 

\chapter{Podsumowanie i wnioski}
\begin{itemize}
\item uzyskane wyniki w świetle postawionych celów i zdefiniowanych wyżej wymagań
\item kierunki ewentualnych danych prac (rozbudowa funkcjonalna …)
\item problemy napotkane w trakcie pracy
\end{itemize}


%%%%%%%%%%%%%%%%%%%%%%%%%%%%
\backmatter 
%\stepcounter{stronyPozaNumeracja}
%\pagenumbering{Roman}
%\setcounter{page}{\value{stronyPozaNumeracja}}
%\pagestyle{tylkoNumeryStron}
%%%%%%%%%%%%%%%%%%%%%%%%%%%%%
%\bibliographystyle{plplain} % bibtex
%\bibliography{bibliografia} % bibtex
\printbibliography           % biblatex 
\addcontentsline{toc}{chapter}{Bibliografia}
%%%%%%%%%%%%%%%%%%%%%%%%%%%%%



\begin{appendices}
 

\chapter{Spis skrótów i symboli}

\begin{itemize}
\item[DNA] kwas deoksyrybonukleinowy (ang. \ang{deoxyribonucleic acid})
\item[MVC] model -- widok -- kontroler (ang. \ang{model--view--controller}) 
\item[$N$] liczebność zbioru danych
\item[$\mu$] stopnień przyleżności do zbioru
\item[$\mathbb{E}$] zbiór krawędzi grafu
\item[$\mathcal{L}$] transformata Laplace'a 
\end{itemize}


\chapter{Źródła}

Jeżeli w pracy konieczne jest umieszczenie długich fragmentów kodu źródłowego, należy je przenieść do załącznika.

\begin{lstlisting}
if (_nClusters < 1)
	throw std::string ("unknown number of clusters");
if (_nIterations < 1 and _epsilon < 0)
	throw std::string ("You should set a maximal number of iteration or minimal difference -- epsilon.");
if (_nIterations > 0 and _epsilon > 0)
	throw std::string ("Both number of iterations and minimal epsilon set -- you should set either number of iterations or minimal epsilon.");
\end{lstlisting}


\begin{minted}[linenos,breaklines,frame=lines]{c++}
if (_nClusters < 1)
   throw std::string ("unknown number of clusters");
if (_nIterations < 1 and _epsilon < 0)
   throw std::string ("You should set a maximal number of iteration or minimal difference -- epsilon.");
if (_nIterations > 0 and _epsilon > 0)
   throw std::string ("Both number of iterations and minimal epsilon set -- you should set either number of iterations or minimal epsilon.");
\end{minted}
 

\chapter{Lista dodatkowych plików, uzupełniających tekst pracy} 


W systemie do pracy dołączono dodatkowe pliki zawierające:
\begin{itemize}
\item źródła programu,
\item dane testowe,
\item film pokazujący działanie opracowanego oprogramowania lub zaprojektowanego i~wykonanego urządzenia,
\item itp.
\end{itemize}

\listoffigures
\addcontentsline{toc}{chapter}{Spis rysunków}
\listoftables
\addcontentsline{toc}{chapter}{Spis tabel}
	
\end{appendices}

\end{document}


%% Finis coronat opus.
