%%%%%%%%%%%%%%%%%%%%%%%%%%%%%%%%%%%%%%%%%%
%                                        %
% Szablon pracy dyplomowej inzynierskiej % 
%                                        %
%%%%%%%%%%%%%%%%%%%%%%%%%%%%%%%%%%%%%%%%%%
%  (c) Krzysztof Simiński, 2018-2021     %
%%%%%%%%%%%%%%%%%%%%%%%%%%%%%%%%%%%%%%%%%%

% kompilacja:

% pdflatex praca
% biber    praca
% pdflatex praca
% pdflatex praca 

%%%%%%%%%%%%%%%%%%%%%%%%%%%%%%%%%%%%%%%%%%



\documentclass[a4paper,twoside,12pt]{book}
\usepackage[utf8]{inputenc}                                      
\usepackage[T1]{fontenc}  
\usepackage{amsmath,amsfonts,amssymb,amsthm}
\usepackage[british,polish]{babel} 
\usepackage{indentfirst}
\usepackage{lmodern}
\usepackage{graphicx} 
\usepackage{hyperref}
\usepackage{booktabs}
%\usepackage{tikz}
%\usepackage{pgfplots}
\usepackage{mathtools}
\usepackage{geometry}
\usepackage[page]{appendix} % toc,
\renewcommand{\appendixtocname}{Dodatki}
\renewcommand{\appendixpagename}{Dodatki}
\renewcommand{\appendixname}{Dodatek}

\usepackage{csquotes}
\usepackage[natbib=true]{biblatex}
\bibliography{bibliografia}



\usepackage{setspace}
\onehalfspacing


\frenchspacing

\usepackage{listings}
%\lstset{
%	language={},
%	basicstyle=\ttfamily,
%	keywordstyle=\lst@ifdisplaystyle\color{blue}\fi,
%	commentstyle=\color{gray}
%}

%%%%%%%%%%%%%%%%%%%%%%%%%%%
% listingi 
\usepackage{listings}
\lstset{%
language=C++,%
commentstyle=\textit,%
identifierstyle=\textsf,%
keywordstyle=\sffamily\bfseries, %\texttt, %
%captionpos=b,%
tabsize=3,%
frame=lines,%
numbers=left,%
numberstyle=\tiny,%
numbersep=5pt,%
breaklines=true,%
morekeywords={descriptor_gaussian,descriptor,partition,fcm_possibilistic,dataset,my_exception,exception,std,vector},%
escapeinside={@*}{*@},%
%texcl=true, % wylacza tryb verbatim w komentarzach jednolinijkowych
}
%%%%%%%%%%%%%%%%%%%%%%%%%%%%%%%%%%%%


%%%%%%%%%

%%%% TODO LIST GENERATOR %%%%%%%%%

%\usepackage{tikz}
%\usepackage{manfnt}   % dangerous sign 
\usepackage{color}
\definecolor{brickred}      {cmyk}{0   , 0.89, 0.94, 0.28}

\makeatletter \newcommand \kslistofremarks{\section*{Uwagi} \@starttoc{rks}}
  \newcommand\l@uwagas[2]
    {\par\noindent \textbf{#2:} %\parbox{10cm}
{#1}\par} \makeatother


\newcommand{\ksremark}[1]{%
{%\marginpar{\textdbend}
{\color{brickred}{[#1]}}}%
\addcontentsline{rks}{uwagas}{\protect{#1}}%
}

\newcommand{\comma}{\ksremark{przecinek}}
\newcommand{\nocomma}{\ksremark{bez przecinka}}
\newcommand{\styl}{\ksremark{styl}}
\newcommand{\ortografia}{\ksremark{ortografia}}
\newcommand{\fleksja}{\ksremark{fleksja}}
\newcommand{\pauza}{\ksremark{pauza `--', nie dywiz `-'}}
\newcommand{\kolokwializm}{\ksremark{kolokwializm}}

%%%%%%%%%%%%%% END OF TODO LIST GENERATOR %%%%%%%%%%%

%%%%%%%%%%%% ZYWA PAGINA %%%%%%%%%%%%%%%
% brak kapitalizacji zywej paginy
\usepackage{fancyhdr}
\pagestyle{fancy}
\fancyhf{}
\fancyhead[LO]{\nouppercase{\it\rightmark}}
\fancyhead[RE]{\nouppercase{\it\leftmark}}
\fancyhead[LE,RO]{\it\thepage}


\fancypagestyle{tylkoNumeryStron}{%
   \fancyhf{} 
   \fancyhead[LE,RO]{\it\thepage}
}

\fancypagestyle{NumeryStronNazwyRozdzialow}{%
   \fancyhf{} 
   \fancyhead[LO]{\nouppercase{\it\rightmark}}
   \fancyhead[RE]{\nouppercase{\it\leftmark}}
   \fancyhead[LE,RO]{\it\thepage}
}


%%%%%%%%%%%%% OBCE WTRETY  
\newcommand{\obcy}[1]{\emph{#1}}
\newcommand{\ang}[1]{{\selectlanguage{british}\obcy{#1}}}
%%%%%%%%%%%%%%%%%%%%%%%%%%%%%

% polskie oznaczenia funkcji matematycznych
\renewcommand{\tan}{\operatorname {tg}}
\renewcommand{\log}{\operatorname {lg}}

% jeszcze jakies drobiazgi

\newcounter{stronyPozaNumeracja}

\newcommand{\hcancel}[1]{%
    \tikz[baseline=(tocancel.base)]{
        \node[inner sep=0pt,outer sep=0pt] (tocancel) {#1};
        \draw[red] (tocancel.south west) -- (tocancel.north east);
    }%
}%

\newcommand{\miesiac}{%
  \ifcase\the\month
  \or styczeń% 1
  \or luty% 2
  \or marzec% 3
  \or kwiecień% 4
  \or maj% 5
  \or czerwiec% 6
  \or lipiec% 7
  \or sierpień% 8
  \or wrzesień% 9
  \or październik% 10
  \or listopad% 11
  \or grudzień% 12
  \fi}


%%%%%%%%%%%%%%%%%%%%%%%%%%%%%%%%%%%%%%%%%%%%%%
% Helvetica font macros for the title page:
\newcommand{\headerfont}{\fontfamily{phv}\fontsize{18}{18}\bfseries\scshape\selectfont}
\newcommand{\titlefont}{\fontfamily{phv}\fontsize{18}{18}\selectfont}
\newcommand{\otherfont}{\fontfamily{phv}\fontsize{14}{14}\selectfont}

%%%%%%%%%%%%%%%%%%%%%%%%%%%%%%%%%%%%%%%%%%%%%%
%%%%%%%%%%%%%%%%%%%%%%%%%%%%%%%%%%%%%%%%%%%%%%
%%%%%%%%%%%%%%%%%%%%%%%%%%%%%%%%%%%%%%%%%%%%%%
%%%%%%%%%%%%%%%%%%%%%%%%%%%%%%%%%%%%%%%%%%%%%%
%%%%%%%%%%%%%%%%%%%%%%%%%%%%%%%%%%%%%%%%%%%%%%
%%%%%%%%%%%%%%%%%%%%%%%%%%%%%%%%%%%%%%%%%%%%%%
%%%%%%%%%%%%%%%%%%%%%%%%%%%%%%%%%%%%%%%%%%%%%%


\newcommand{\autor}{Imię Nazwisko}
\newcommand{\promotor}{dr inż. Imię Nazwisko}
\newcommand{\konsultant}{dr inż. Imię Nazwisko}
\newcommand{\tytul}{Tytuł pracy dyplomowej inżynierskiej}
\newcommand{\polsl}{Politechnika Śląska}
\newcommand{\wydzial}{Wydział Automatyki, Elektroniki i Informatyki}
\newcommand{\kierunek}{Kierunek: [wpisać właściwy]}


\begin{document}
%\kslistofremarks 
	
%%%%%%%%%%%%%%%%%%  STRONA TYTULOWA %%%%%%%%%%%%%%%%%%%
\pagestyle{empty}
{
	\newgeometry{top=2.5cm,%
	             bottom=2.5cm,%
	             left=3cm,
	             right=2.5cm}
	\sffamily
	\rule{0cm}{0cm}
	
	\begin{center}
	\includegraphics[width=29mm]{logo_pl.jpg}
	\end{center} 
	\vspace{1cm}
	\begin{center}
	\headerfont \polsl
	\end{center}
	\begin{center}
	\headerfont \wydzial
	\end{center}
	\begin{center}
	\headerfont \kierunek
	\end{center}
	\vfill
	\begin{center}
	\titlefont Praca dyplomowa inżynierska
	\end{center}
	\vfill
	
	\begin{center}
	\otherfont \tytul\par
	\end{center}
	
	\vfill
	
	\vfill
	 
	\noindent\vbox
	{
		\hbox{\otherfont autor: \autor}
		\vspace{12pt}
		\hbox{\otherfont kierujący pracą: \promotor}
		\vspace{12pt}  % zakomentuj, jezeli nie ma konsultanta
		\hbox{\otherfont konsultant: \konsultant} % zakomentuj, jezeli nie ma konsultanta
	}
	\vfill 
 
   \begin{center}
   \otherfont Gliwice,  \miesiac\ \the\year
   \end{center}	
	\restoregeometry
}
  

\cleardoublepage
 

\rmfamily
\normalfont


%%%%%%%%%%%%%%%%%% SPIS TRESCI %%%%%%%%%%%%%%%%%%%%%%
\pagenumbering{Roman}
\pagestyle{tylkoNumeryStron}
\tableofcontents

%%%%%%%%%%%%%%%%%%%%%%%%%%%%%%%%%%%%%%%%%%%%%%%%%%%%%
\setcounter{stronyPozaNumeracja}{\value{page}}
\mainmatter
\pagestyle{empty}

\chapter*{Streszczenie}
\addcontentsline{toc}{chapter}{Streszczenie}

Odpowiednie pole w systemie APD powinno zawierać kopię tego streszczenia. Streszczenie wraz ze słowami kluczowymi nie powinno przekroczyć jednej strony.

\paragraph{Słowa kluczowe:} 2-5 slow (fraz) kluczowych, oddzielonych przecinkami



\cleardoublepage

\pagestyle{NumeryStronNazwyRozdzialow}

%%%%%%%%%%%%%% wlasciwa tresc pracy %%%%%%%%%%%%%%%%%

\chapter{Wstęp}

\begin{itemize}
\item wprowadzenie w problem/zagadnienie
\item osadzenie problemu w dziedzinie
\item cel pracy
\item zakres pracy
\item zwięzła charakterystyka rozdziałów
\item jednoznaczne określenie wkładu autora, w przypadku prac wieloosobowych – tabela z autorstwem poszczególnych elementów pracy
\end{itemize}


\chapter{[Analiza tematu]}

\begin{itemize}
\item sformułowanie problemu
\item osadzenie tematu w kontekście aktualnego stanu wiedzy (\ang{state of the art}) o poruszanym problemie
\item  studia literaturowe \cite{bib:artykul,bib:ksiazka,bib:konferencja,bib:internet} -  opis znanych rozwiązań (także opisanych naukowo, jeżeli problem jest poruszany w publikacjach naukowych), algorytmów, 
\end{itemize}




\chapter{Wymagania i narzędzia}

\begin{itemize}
\item wymagania funkcjonalne i niefunkcjonalne
\item przypadki użycia (diagramy UML) -- dla prac, w których mają zastosowanie
\item opis narzędzi, metod eksperymentalnych, metod modelowania itp.
\item metodyka pracy nad projektowaniem i implementacją -- dla prac, w których ma to zastosowanie
\end{itemize}

\chapter{[Właściwy dla kierunku -- np. Specyfikacja zewnętrzna]}
Jeśli „Specyfikacja zewnętrzna”:
\begin{itemize}
\item  wymagania sprzętowe i programowe
\item  sposób instalacji
\item  sposób aktywacji
\item  kategorie użytkowników
\item  sposób obsługi
\item  administracja systemem
\item  kwestie bezpieczeństwa
\item  przykład działania
\item  scenariusze korzystania z systemu (ilustrowane zrzutami z ekranu lub generowanymi dokumentami)
\end{itemize}

\begin{figure}
\centering
\includegraphics[width=10cm]{logo_pl.jpg}
\caption{Podpis rysunku po rysunkiem.}
\label{fig:2}
\end{figure}


\chapter{[Właściwy dla kierunku -- np. Specyfikacja wewnętrzna]}
Jeśli „Specyfikacja wewnętrzna”:
\begin{itemize}
\item przedstawienie idei
\item architektura systemu
\item opis struktur danych (i organizacji baz danych)
\item komponenty, moduły, biblioteki, przegląd ważniejszych klas (jeśli występują)
\item przegląd ważniejszych algorytmów (jeśli występują)
\item szczegóły implementacji wybranych fragmentów, zastosowane wzorce projektowe
\item diagramy UML
\end{itemize}



Krótka wstawka kodu w linii tekstu jest możliwa, np. \lstinline|descriptor|, a nawet \lstinline|descriptor_gaussian|. 
Dłuższe fragmenty lepiej jest umieszczać jako rysunek, np. kod na rysunku \ref{fig:pseudokod}, a naprawdę długie fragmenty – w załączniku.

\begin{figure}
\centering
\begin{lstlisting}
class descriptor_gaussian : virtual public descriptor
{
   protected:
      /** core of the gaussian fuzzy set */
      double _mean;
      /** fuzzyfication of the gaussian fuzzy set */
      double _stddev;
      
   public:
      /** @param mean core of the set
          @param stddev standard deviation */
      descriptor_gaussian (double mean, double stddev);
      descriptor_gaussian (const descriptor_gaussian & w);
      virtual ~descriptor_gaussian();
      virtual descriptor * clone () const;
      
      /** The method elaborates membership to the gaussian fuzzy set. */
      virtual double getMembership (double x) const;
     
};
\end{lstlisting}
\caption{Klasa \lstinline|descriptor_gaussian|.}
\label{fig:pseudokod}
\end{figure}


\chapter{Weryfikacja i walidacja}
\begin{itemize}
\item sposób testowania w ramach pracy (np. odniesienie do modelu V)
\item organizacja eksperymentów
\item przypadki testowe zakres testowania (pełny/niepełny)
\item wykryte i usunięte błędy
\item opcjonalnie wyniki badań eksperymentalnych
\end{itemize}

\begin{table}
\centering
\caption{Opis tabeli nad nią.}
\label{id:tab:wyniki}
\begin{tabular}{rrrrrrrr}
\toprule
	         &                                     \multicolumn{7}{c}{metoda}                                      \\
	         \cmidrule{2-8}
	         &         &         &        \multicolumn{3}{c}{alg. 3}        & \multicolumn{2}{c}{alg. 4, $\gamma = 2$} \\
	         \cmidrule(r){4-6}\cmidrule(r){7-8}
	$\zeta$ &     alg. 1 &   alg. 2 & $\alpha= 1.5$ & $\alpha= 2$ & $\alpha= 3$ &   $\beta = 0.1$  &   $\beta = -0.1$ \\
\midrule
	       0 &  8.3250 & 1.45305 &       7.5791 &    14.8517 &    20.0028 & 1.16396 &                       1.1365 \\
	       5 &  0.6111 & 2.27126 &       6.9952 &    13.8560 &    18.6064 & 1.18659 &                       1.1630 \\
	      10 & 11.6126 & 2.69218 &       6.2520 &    12.5202 &    16.8278 & 1.23180 &                       1.2045 \\
	      15 &  0.5665 & 2.95046 &       5.7753 &    11.4588 &    15.4837 & 1.25131 &                       1.2614 \\
	      20 & 15.8728 & 3.07225 &       5.3071 &    10.3935 &    13.8738 & 1.25307 &                       1.2217 \\
	      25 &  0.9791 & 3.19034 &       5.4575 &     9.9533 &    13.0721 & 1.27104 &                       1.2640 \\
	      30 &  2.0228 & 3.27474 &       5.7461 &     9.7164 &    12.2637 & 1.33404 &                       1.3209 \\
	      35 & 13.4210 & 3.36086 &       6.6735 &    10.0442 &    12.0270 & 1.35385 &                       1.3059 \\
	      40 & 13.2226 & 3.36420 &       7.7248 &    10.4495 &    12.0379 & 1.34919 &                       1.2768 \\
	      45 & 12.8445 & 3.47436 &       8.5539 &    10.8552 &    12.2773 & 1.42303 &                       1.4362 \\
	      50 & 12.9245 & 3.58228 &       9.2702 &    11.2183 &    12.3990 & 1.40922 &                       1.3724 \\
\bottomrule
\end{tabular}
\end{table}  
 

\chapter{Podsumowanie i wnioski}
\begin{itemize}
\item uzyskane wyniki w świetle postawionych celów i zdefiniowanych wyżej wymagań
\item kierunki ewentualnych danych prac (rozbudowa funkcjonalna …)
\item problemy napotkane w trakcie pracy
\end{itemize}


%%%%%%%%%%%%%%%%%%%%%%%%%%%%
\backmatter 
\stepcounter{stronyPozaNumeracja}
\pagenumbering{Roman}
\setcounter{page}{\value{stronyPozaNumeracja}}
\pagestyle{tylkoNumeryStron}
%%%%%%%%%%%%%%%%%%%%%%%%%%%%%
%\bibliographystyle{plplain} % bibtex
%\bibliography{bibliografia} % bibtex
\printbibliography           % biblatex 
\addcontentsline{toc}{chapter}{Bibliografia}
%%%%%%%%%%%%%%%%%%%%%%%%%%%%%



\begin{appendices}
 

\chapter*{Spis skrótów i symboli}
\addcontentsline{toc}{chapter}{Spis skrótów i symboli}

\begin{itemize}
\item[DNA] kwas deoksyrybonukleinowy (ang. \ang{deoxyribonucleic acid})
\item[MVC] model -- widok -- kontroler (ang. \ang{model--view--controller}) 
\item[$N$] liczebność zbioru danych
\item[$\mu$] stopnień przyleżności do zbioru
\item[$\mathbb{E}$] zbiór krawędzi grafu
\item[$\mathcal{L}$] transformata Laplace'a 
\end{itemize}


\chapter*{Źródła}
\addcontentsline{toc}{chapter}{Źródła}

Jeżeli w pracy konieczne jest umieszczenie długich fragmentów kodu źródłowego, należy je przenieść do załącznika.

\begin{lstlisting}
partition fcm_possibilistic::doPartition
                             (const dataset & ds)
{
   try
   {
      if (_nClusters < 1)
         throw std::string ("unknown number of clusters");
      if (_nIterations < 1 and _epsilon < 0)
         throw std::string ("You should set a maximal number of iteration or minimal difference -- epsilon.");
      if (_nIterations > 0 and _epsilon > 0)
         throw std::string ("Both number of iterations and minimal epsilon set -- you should set either number of iterations or minimal epsilon.");
   
      auto mX = ds.getMatrix();
      std::size_t nAttr = ds.getNumberOfAttributes();
      std::size_t nX    = ds.getNumberOfData();
      std::vector<std::vector<double>> mV;
      mU = std::vector<std::vector<double>> (_nClusters);
      for (auto & u : mU)
         u = std::vector<double> (nX);
      randomise(mU);
      normaliseByColumns(mU);
      calculateEtas(_nClusters, nX, ds);
      if (_nIterations > 0)
      {
         for (int iter = 0; iter < _nIterations; iter++)
         {
            mV = calculateClusterCentres(mU, mX);
            mU = modifyPartitionMatrix (mV, mX);
         }
      }
      else if (_epsilon > 0)
      {
         double frob;
         do 
         {
            mV = calculateClusterCentres(mU, mX);
            auto mUnew = modifyPartitionMatrix (mV, mX);
            
            frob = Frobenius_norm_of_difference (mU, mUnew);
            mU = mUnew;
         } while (frob > _epsilon);
      }
      mV = calculateClusterCentres(mU, mX);
      std::vector<std::vector<double>> mS = calculateClusterFuzzification(mU, mV, mX);
      
      partition part;
      for (int c = 0; c < _nClusters; c++)
      {
         cluster cl; 
         for (std::size_t a = 0; a < nAttr; a++)
         {
            descriptor_gaussian d (mV[c][a], mS[c][a]);
            cl.addDescriptor(d);
         }
         part.addCluster(cl);
      }
      return part;
   }
   catch (my_exception & ex)                                  
   {                                                       
      throw my_exception (__FILE__, __FUNCTION__, __LINE__, ex.what()); 
   }                                                          
   catch (std::exception & ex)                                 
   {                                                            
      throw my_exceptionn (__FILE__, __FUNCTION__, __LINE__, ex.what()); 
   }                                                            
   catch (std::string & ex)                                     
   {                                                            
      throw my_exception (__FILE__, __FUNCTION__, __LINE__, ex);        
   }                                                             
   catch (...)                                                   
   {                                                             
      throw my_exception (__FILE__, __FUNCTION__, __LINE__, "unknown expection");       
   }  
}
\end{lstlisting}
 

\chapter*{Lista plików dodatkowych, uzupełniających tekst pracy}
\addcontentsline{toc}{chapter}{Lista plików dodatkowych}

W systemie do pracy dołączono dodatkowe pliki zawierające:
\begin{itemize}
\item źródła programu,
\item dane testowe,
\item film pokazujący działanie opracowanego oprogramowania lub zaprojektowanego i wykonanego urządzenia,
\item itp.
\end{itemize}

\listoffigures
\addcontentsline{toc}{chapter}{Spis rysunków}
\listoftables
\addcontentsline{toc}{chapter}{Spis tabel}
	
\end{appendices}


\end{document}


%% Finis coronat opus.
